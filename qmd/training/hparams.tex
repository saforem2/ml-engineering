% Options for packages loaded elsewhere
\PassOptionsToPackage{unicode}{hyperref}
\PassOptionsToPackage{hyphens}{url}
\PassOptionsToPackage{dvipsnames,svgnames,x11names}{xcolor}
%
\documentclass[
]{report}

\usepackage{amsmath,amssymb}
\usepackage{iftex}
\ifPDFTeX
  \usepackage[T1]{fontenc}
  \usepackage[utf8]{inputenc}
  \usepackage{textcomp} % provide euro and other symbols
\else % if luatex or xetex
  \usepackage{unicode-math}
  \defaultfontfeatures{Scale=MatchLowercase}
  \defaultfontfeatures[\rmfamily]{Ligatures=TeX,Scale=1}
\fi
\usepackage{lmodern}
\ifPDFTeX\else  
    % xetex/luatex font selection
\fi
% Use upquote if available, for straight quotes in verbatim environments
\IfFileExists{upquote.sty}{\usepackage{upquote}}{}
\IfFileExists{microtype.sty}{% use microtype if available
  \usepackage[]{microtype}
  \UseMicrotypeSet[protrusion]{basicmath} % disable protrusion for tt fonts
}{}
\makeatletter
\@ifundefined{KOMAClassName}{% if non-KOMA class
  \IfFileExists{parskip.sty}{%
    \usepackage{parskip}
  }{% else
    \setlength{\parindent}{0pt}
    \setlength{\parskip}{6pt plus 2pt minus 1pt}}
}{% if KOMA class
  \KOMAoptions{parskip=half}}
\makeatother
\usepackage{xcolor}
\setlength{\emergencystretch}{3em} % prevent overfull lines
\setcounter{secnumdepth}{-\maxdimen} % remove section numbering
% Make \paragraph and \subparagraph free-standing
\ifx\paragraph\undefined\else
  \let\oldparagraph\paragraph
  \renewcommand{\paragraph}[1]{\oldparagraph{#1}\mbox{}}
\fi
\ifx\subparagraph\undefined\else
  \let\oldsubparagraph\subparagraph
  \renewcommand{\subparagraph}[1]{\oldsubparagraph{#1}\mbox{}}
\fi


\providecommand{\tightlist}{%
  \setlength{\itemsep}{0pt}\setlength{\parskip}{0pt}}\usepackage{longtable,booktabs,array}
\usepackage{calc} % for calculating minipage widths
% Correct order of tables after \paragraph or \subparagraph
\usepackage{etoolbox}
\makeatletter
\patchcmd\longtable{\par}{\if@noskipsec\mbox{}\fi\par}{}{}
\makeatother
% Allow footnotes in longtable head/foot
\IfFileExists{footnotehyper.sty}{\usepackage{footnotehyper}}{\usepackage{footnote}}
\makesavenoteenv{longtable}
\usepackage{graphicx}
\makeatletter
\def\maxwidth{\ifdim\Gin@nat@width>\linewidth\linewidth\else\Gin@nat@width\fi}
\def\maxheight{\ifdim\Gin@nat@height>\textheight\textheight\else\Gin@nat@height\fi}
\makeatother
% Scale images if necessary, so that they will not overflow the page
% margins by default, and it is still possible to overwrite the defaults
% using explicit options in \includegraphics[width, height, ...]{}
\setkeys{Gin}{width=\maxwidth,height=\maxheight,keepaspectratio}
% Set default figure placement to htbp
\makeatletter
\def\fps@figure{htbp}
\makeatother

\makeatletter
\@ifpackageloaded{caption}{}{\usepackage{caption}}
\AtBeginDocument{%
\ifdefined\contentsname
  \renewcommand*\contentsname{Table of contents}
\else
  \newcommand\contentsname{Table of contents}
\fi
\ifdefined\listfigurename
  \renewcommand*\listfigurename{List of Figures}
\else
  \newcommand\listfigurename{List of Figures}
\fi
\ifdefined\listtablename
  \renewcommand*\listtablename{List of Tables}
\else
  \newcommand\listtablename{List of Tables}
\fi
\ifdefined\figurename
  \renewcommand*\figurename{Figure}
\else
  \newcommand\figurename{Figure}
\fi
\ifdefined\tablename
  \renewcommand*\tablename{Table}
\else
  \newcommand\tablename{Table}
\fi
}
\@ifpackageloaded{float}{}{\usepackage{float}}
\floatstyle{ruled}
\@ifundefined{c@chapter}{\newfloat{codelisting}{h}{lop}}{\newfloat{codelisting}{h}{lop}[chapter]}
\floatname{codelisting}{Listing}
\newcommand*\listoflistings{\listof{codelisting}{List of Listings}}
\makeatother
\makeatletter
\makeatother
\makeatletter
\@ifpackageloaded{caption}{}{\usepackage{caption}}
\@ifpackageloaded{subcaption}{}{\usepackage{subcaption}}
\makeatother
\ifLuaTeX
  \usepackage{selnolig}  % disable illegal ligatures
\fi
\usepackage{bookmark}

\IfFileExists{xurl.sty}{\usepackage{xurl}}{} % add URL line breaks if available
\urlstyle{same} % disable monospaced font for URLs
\hypersetup{
  colorlinks=true,
  linkcolor={blue},
  filecolor={Maroon},
  citecolor={Blue},
  urlcolor={Blue},
  pdfcreator={LaTeX via pandoc}}

\author{}
\date{2024-02-13}

\begin{document}

\chapter{Selecting Training Hyper-Parameters And Model
Initializations}\label{selecting-training-hyper-parameters-and-model-initializations}

The easiest way to find a good hparam and model init starter set is to
steal it from a similar training that you know has succeeded. Here is a
\href{../resources/README.md\#publicly-available-training-llmvlm-logbooks}{collection
of public training LLM/VLM logbooks} to get you started. The other
common source is papers if they disclose that information. You can also
try to reach out to the authors and ask them for these details if they
didn't publish it.

\section{Glossary}\label{glossary}

Training jargon uses a multitude of abbreviations and terms, so here are
some important for this chapter.

\begin{itemize}
\tightlist
\item
  BS: Batch Size - here we mean batch size per gpu, often it is also
  referred to as MBS (micro-batch-size)
\item
  GBS: Global Batch Size - total batch size per iteration - may include
  gradient accumulation
\item
  GAS: Gradient Accumulation Steps - how many forward/backward cycles to
  perform before one full iteration is complete
\item
  TFLOPs: Trillion FLOPs per second -
  \href{https://en.wikipedia.org/wiki/FLOPS}{FLOPS}
\item
  PP: Pipeline Parallelism
\end{itemize}

\section{Global Batch Size Ramp Up}\label{global-batch-size-ramp-up}

If you intend to train with a very large GBS, with say 1024, or 2048
samples and even higher, when you just start training, it's very
wasteful to feed such large batch sizes to the model. At this point it's
totally random and can't benefit from having too refined data. Therefore
to save data and resources, one often ramps up the global batch size
over some period of time.

It's also important to not start with GBS that is too small, since
otherwise the progress won't be efficient. When there is too little data
the compute (TFLOPS) is inefficient and will slow everything down. This
is especially so when Pipeline Parallelism (PP) is used, since the most
important thing about PP tuneup is a small GPU idleness bubble, and the
smaller the GBS the larger the bubble is.

For example, for BLOOM-176B, where we did use PP, after doing throughput
benchmarking we found that starting with GBS=16 was incredibly slow (8
TFLOPs), so we eventually started with GBS=192 (73 TFLOPs) and then we
ramped up to GBS=2048 (150 TFLOPs) - we increased GBS by 16 every
9\_765\_625 samples.

\subsection{STD Init}\label{std-init}

This hyper parameter is super-important and it requires math to get it
right. For details see \href{instabilities\#std-init}{STD Init}.



\end{document}
