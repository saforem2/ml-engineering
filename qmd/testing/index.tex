% Options for packages loaded elsewhere
\PassOptionsToPackage{unicode}{hyperref}
\PassOptionsToPackage{hyphens}{url}
\PassOptionsToPackage{dvipsnames,svgnames,x11names}{xcolor}
%
\documentclass[
]{report}

\usepackage{amsmath,amssymb}
\usepackage{iftex}
\ifPDFTeX
  \usepackage[T1]{fontenc}
  \usepackage[utf8]{inputenc}
  \usepackage{textcomp} % provide euro and other symbols
\else % if luatex or xetex
  \usepackage{unicode-math}
  \defaultfontfeatures{Scale=MatchLowercase}
  \defaultfontfeatures[\rmfamily]{Ligatures=TeX,Scale=1}
\fi
\usepackage{lmodern}
\ifPDFTeX\else  
    % xetex/luatex font selection
\fi
% Use upquote if available, for straight quotes in verbatim environments
\IfFileExists{upquote.sty}{\usepackage{upquote}}{}
\IfFileExists{microtype.sty}{% use microtype if available
  \usepackage[]{microtype}
  \UseMicrotypeSet[protrusion]{basicmath} % disable protrusion for tt fonts
}{}
\makeatletter
\@ifundefined{KOMAClassName}{% if non-KOMA class
  \IfFileExists{parskip.sty}{%
    \usepackage{parskip}
  }{% else
    \setlength{\parindent}{0pt}
    \setlength{\parskip}{6pt plus 2pt minus 1pt}}
}{% if KOMA class
  \KOMAoptions{parskip=half}}
\makeatother
\usepackage{xcolor}
\setlength{\emergencystretch}{3em} % prevent overfull lines
\setcounter{secnumdepth}{-\maxdimen} % remove section numbering
% Make \paragraph and \subparagraph free-standing
\ifx\paragraph\undefined\else
  \let\oldparagraph\paragraph
  \renewcommand{\paragraph}[1]{\oldparagraph{#1}\mbox{}}
\fi
\ifx\subparagraph\undefined\else
  \let\oldsubparagraph\subparagraph
  \renewcommand{\subparagraph}[1]{\oldsubparagraph{#1}\mbox{}}
\fi

\usepackage{color}
\usepackage{fancyvrb}
\newcommand{\VerbBar}{|}
\newcommand{\VERB}{\Verb[commandchars=\\\{\}]}
\DefineVerbatimEnvironment{Highlighting}{Verbatim}{commandchars=\\\{\}}
% Add ',fontsize=\small' for more characters per line
\usepackage{framed}
\definecolor{shadecolor}{RGB}{241,243,245}
\newenvironment{Shaded}{\begin{snugshade}}{\end{snugshade}}
\newcommand{\AlertTok}[1]{\textcolor[rgb]{0.68,0.00,0.00}{#1}}
\newcommand{\AnnotationTok}[1]{\textcolor[rgb]{0.37,0.37,0.37}{#1}}
\newcommand{\AttributeTok}[1]{\textcolor[rgb]{0.40,0.45,0.13}{#1}}
\newcommand{\BaseNTok}[1]{\textcolor[rgb]{0.68,0.00,0.00}{#1}}
\newcommand{\BuiltInTok}[1]{\textcolor[rgb]{0.00,0.23,0.31}{#1}}
\newcommand{\CharTok}[1]{\textcolor[rgb]{0.13,0.47,0.30}{#1}}
\newcommand{\CommentTok}[1]{\textcolor[rgb]{0.37,0.37,0.37}{#1}}
\newcommand{\CommentVarTok}[1]{\textcolor[rgb]{0.37,0.37,0.37}{\textit{#1}}}
\newcommand{\ConstantTok}[1]{\textcolor[rgb]{0.56,0.35,0.01}{#1}}
\newcommand{\ControlFlowTok}[1]{\textcolor[rgb]{0.00,0.23,0.31}{#1}}
\newcommand{\DataTypeTok}[1]{\textcolor[rgb]{0.68,0.00,0.00}{#1}}
\newcommand{\DecValTok}[1]{\textcolor[rgb]{0.68,0.00,0.00}{#1}}
\newcommand{\DocumentationTok}[1]{\textcolor[rgb]{0.37,0.37,0.37}{\textit{#1}}}
\newcommand{\ErrorTok}[1]{\textcolor[rgb]{0.68,0.00,0.00}{#1}}
\newcommand{\ExtensionTok}[1]{\textcolor[rgb]{0.00,0.23,0.31}{#1}}
\newcommand{\FloatTok}[1]{\textcolor[rgb]{0.68,0.00,0.00}{#1}}
\newcommand{\FunctionTok}[1]{\textcolor[rgb]{0.28,0.35,0.67}{#1}}
\newcommand{\ImportTok}[1]{\textcolor[rgb]{0.00,0.46,0.62}{#1}}
\newcommand{\InformationTok}[1]{\textcolor[rgb]{0.37,0.37,0.37}{#1}}
\newcommand{\KeywordTok}[1]{\textcolor[rgb]{0.00,0.23,0.31}{#1}}
\newcommand{\NormalTok}[1]{\textcolor[rgb]{0.00,0.23,0.31}{#1}}
\newcommand{\OperatorTok}[1]{\textcolor[rgb]{0.37,0.37,0.37}{#1}}
\newcommand{\OtherTok}[1]{\textcolor[rgb]{0.00,0.23,0.31}{#1}}
\newcommand{\PreprocessorTok}[1]{\textcolor[rgb]{0.68,0.00,0.00}{#1}}
\newcommand{\RegionMarkerTok}[1]{\textcolor[rgb]{0.00,0.23,0.31}{#1}}
\newcommand{\SpecialCharTok}[1]{\textcolor[rgb]{0.37,0.37,0.37}{#1}}
\newcommand{\SpecialStringTok}[1]{\textcolor[rgb]{0.13,0.47,0.30}{#1}}
\newcommand{\StringTok}[1]{\textcolor[rgb]{0.13,0.47,0.30}{#1}}
\newcommand{\VariableTok}[1]{\textcolor[rgb]{0.07,0.07,0.07}{#1}}
\newcommand{\VerbatimStringTok}[1]{\textcolor[rgb]{0.13,0.47,0.30}{#1}}
\newcommand{\WarningTok}[1]{\textcolor[rgb]{0.37,0.37,0.37}{\textit{#1}}}

\providecommand{\tightlist}{%
  \setlength{\itemsep}{0pt}\setlength{\parskip}{0pt}}\usepackage{longtable,booktabs,array}
\usepackage{calc} % for calculating minipage widths
% Correct order of tables after \paragraph or \subparagraph
\usepackage{etoolbox}
\makeatletter
\patchcmd\longtable{\par}{\if@noskipsec\mbox{}\fi\par}{}{}
\makeatother
% Allow footnotes in longtable head/foot
\IfFileExists{footnotehyper.sty}{\usepackage{footnotehyper}}{\usepackage{footnote}}
\makesavenoteenv{longtable}
\usepackage{graphicx}
\makeatletter
\def\maxwidth{\ifdim\Gin@nat@width>\linewidth\linewidth\else\Gin@nat@width\fi}
\def\maxheight{\ifdim\Gin@nat@height>\textheight\textheight\else\Gin@nat@height\fi}
\makeatother
% Scale images if necessary, so that they will not overflow the page
% margins by default, and it is still possible to overwrite the defaults
% using explicit options in \includegraphics[width, height, ...]{}
\setkeys{Gin}{width=\maxwidth,height=\maxheight,keepaspectratio}
% Set default figure placement to htbp
\makeatletter
\def\fps@figure{htbp}
\makeatother

\makeatletter
\@ifpackageloaded{caption}{}{\usepackage{caption}}
\AtBeginDocument{%
\ifdefined\contentsname
  \renewcommand*\contentsname{Table of contents}
\else
  \newcommand\contentsname{Table of contents}
\fi
\ifdefined\listfigurename
  \renewcommand*\listfigurename{List of Figures}
\else
  \newcommand\listfigurename{List of Figures}
\fi
\ifdefined\listtablename
  \renewcommand*\listtablename{List of Tables}
\else
  \newcommand\listtablename{List of Tables}
\fi
\ifdefined\figurename
  \renewcommand*\figurename{Figure}
\else
  \newcommand\figurename{Figure}
\fi
\ifdefined\tablename
  \renewcommand*\tablename{Table}
\else
  \newcommand\tablename{Table}
\fi
}
\@ifpackageloaded{float}{}{\usepackage{float}}
\floatstyle{ruled}
\@ifundefined{c@chapter}{\newfloat{codelisting}{h}{lop}}{\newfloat{codelisting}{h}{lop}[chapter]}
\floatname{codelisting}{Listing}
\newcommand*\listoflistings{\listof{codelisting}{List of Listings}}
\makeatother
\makeatletter
\makeatother
\makeatletter
\@ifpackageloaded{caption}{}{\usepackage{caption}}
\@ifpackageloaded{subcaption}{}{\usepackage{subcaption}}
\makeatother
\ifLuaTeX
  \usepackage{selnolig}  % disable illegal ligatures
\fi
\usepackage{bookmark}

\IfFileExists{xurl.sty}{\usepackage{xurl}}{} % add URL line breaks if available
\urlstyle{same} % disable monospaced font for URLs
\hypersetup{
  pdftitle={✏️ Testing},
  colorlinks=true,
  linkcolor={blue},
  filecolor={Maroon},
  citecolor={Blue},
  urlcolor={Blue},
  pdfcreator={LaTeX via pandoc}}

\title{✏️ Testing}
\author{}
\date{2024-02-20}

\begin{document}
\maketitle

\chapter{Writing and Running Tests}\label{writing-and-running-tests}

Note: a part of this document refers to functionality provided by the
included \url{testing_utils.py}, the bulk of which I have developed
while I worked at HuggingFace.

This document covers both \texttt{pytest} and \texttt{unittest}
functionalities and shows how both can be used together.

\section{Running tests}\label{running-tests}

\subsection{Run all tests}\label{run-all-tests}

\begin{Shaded}
\begin{Highlighting}[]
\ExtensionTok{pytest}
\end{Highlighting}
\end{Shaded}

I use the following alias:

\begin{Shaded}
\begin{Highlighting}[]
\BuiltInTok{alias}\NormalTok{ pyt=}\StringTok{"pytest {-}{-}disable{-}warnings {-}{-}instafail {-}rA"}
\end{Highlighting}
\end{Shaded}

which tells pytest to:

\begin{itemize}
\tightlist
\item
  disable warning
\item
  \texttt{-\/-instafail} shows failures as they happen, and not at the
  end
\item
  \texttt{-rA} generates a short test summary info
\end{itemize}

it requires you to install:

\begin{Shaded}
\begin{Highlighting}[]
\ExtensionTok{pip}\NormalTok{ install pytest{-}instafail}
\end{Highlighting}
\end{Shaded}

\subsection{Getting the list of all
tests}\label{getting-the-list-of-all-tests}

Show all tests in the test suite:

\begin{Shaded}
\begin{Highlighting}[]
\ExtensionTok{pytest} \AttributeTok{{-}{-}collect{-}only} \AttributeTok{{-}q}
\end{Highlighting}
\end{Shaded}

Show all tests in a given test file:

\begin{Shaded}
\begin{Highlighting}[]
\ExtensionTok{pytest}\NormalTok{ tests/test\_optimization.py }\AttributeTok{{-}{-}collect{-}only} \AttributeTok{{-}q}
\end{Highlighting}
\end{Shaded}

I use the following alias:

\begin{Shaded}
\begin{Highlighting}[]
\BuiltInTok{alias}\NormalTok{ pytc=}\StringTok{"pytest {-}{-}disable{-}warnings {-}{-}collect{-}only {-}q"}
\end{Highlighting}
\end{Shaded}

\subsection{Run a specific test
module}\label{run-a-specific-test-module}

To run an individual test module:

\begin{Shaded}
\begin{Highlighting}[]
\ExtensionTok{pytest}\NormalTok{ tests/utils/test\_logging.py}
\end{Highlighting}
\end{Shaded}

\subsection{Run specific tests}\label{run-specific-tests}

If \texttt{unittest} is used, to run specific subtests you need to know
the name of the \texttt{unittest} class containing those tests. For
example, it could be:

\begin{Shaded}
\begin{Highlighting}[]
\ExtensionTok{pytest}\NormalTok{ tests/test\_optimization.py::OptimizationTest::test\_adam\_w}
\end{Highlighting}
\end{Shaded}

Here:

\begin{itemize}
\tightlist
\item
  \texttt{tests/test\_optimization.py} - the file with tests
\item
  \texttt{OptimizationTest} - the name of the test class
\item
  \texttt{test\_adam\_w} - the name of the specific test function
\end{itemize}

If the file contains multiple classes, you can choose to run only tests
of a given class. For example:

\begin{Shaded}
\begin{Highlighting}[]
\ExtensionTok{pytest}\NormalTok{ tests/test\_optimization.py::OptimizationTest}
\end{Highlighting}
\end{Shaded}

will run all the tests inside that class.

As mentioned earlier you can see what tests are contained inside the
\texttt{OptimizationTest} class by running:

\begin{Shaded}
\begin{Highlighting}[]
\ExtensionTok{pytest}\NormalTok{ tests/test\_optimization.py::OptimizationTest }\AttributeTok{{-}{-}collect{-}only} \AttributeTok{{-}q}
\end{Highlighting}
\end{Shaded}

You can run tests by keyword expressions.

To run only tests whose name contains \texttt{adam}:

\begin{Shaded}
\begin{Highlighting}[]
\ExtensionTok{pytest} \AttributeTok{{-}k}\NormalTok{ adam tests/test\_optimization.py}
\end{Highlighting}
\end{Shaded}

Logical \texttt{and} and \texttt{or} can be used to indicate whether all
keywords should match or either. \texttt{not} can be used to negate.

To run all tests except those whose name contains \texttt{adam}:

\begin{Shaded}
\begin{Highlighting}[]
\ExtensionTok{pytest} \AttributeTok{{-}k} \StringTok{"not adam"}\NormalTok{ tests/test\_optimization.py}
\end{Highlighting}
\end{Shaded}

And you can combine the two patterns in one:

\begin{Shaded}
\begin{Highlighting}[]
\ExtensionTok{pytest} \AttributeTok{{-}k} \StringTok{"ada and not adam"}\NormalTok{ tests/test\_optimization.py}
\end{Highlighting}
\end{Shaded}

For example to run both \texttt{test\_adafactor} and
\texttt{test\_adam\_w} you can use:

\begin{Shaded}
\begin{Highlighting}[]
\ExtensionTok{pytest} \AttributeTok{{-}k} \StringTok{"test\_adam\_w or test\_adam\_w"}\NormalTok{ tests/test\_optimization.py}
\end{Highlighting}
\end{Shaded}

Note that we use \texttt{or} here, since we want either of the keywords
to match to include both.

If you want to include only tests that include both patterns,
\texttt{and} is to be used:

\begin{Shaded}
\begin{Highlighting}[]
\ExtensionTok{pytest} \AttributeTok{{-}k} \StringTok{"test and ada"}\NormalTok{ tests/test\_optimization.py}
\end{Highlighting}
\end{Shaded}

\subsection{Run only modified tests}\label{run-only-modified-tests}

You can run the tests related to the unstaged files or the current
branch (according to Git) by using
\href{https://github.com/anapaulagomes/pytest-picked}{pytest-picked}.
This is a great way of quickly testing your changes didn't break
anything, since it won't run the tests related to files you didn't
touch.

\begin{Shaded}
\begin{Highlighting}[]
\ExtensionTok{pip}\NormalTok{ install pytest{-}picked}
\end{Highlighting}
\end{Shaded}

\begin{Shaded}
\begin{Highlighting}[]
\ExtensionTok{pytest} \AttributeTok{{-}{-}picked}
\end{Highlighting}
\end{Shaded}

All tests will be run from files and folders which are modified, but not
yet committed.

\subsection{Automatically rerun failed tests on source
modification}\label{automatically-rerun-failed-tests-on-source-modification}

\href{https://github.com/pytest-dev/pytest-xdist}{pytest-xdist} provides
a very useful feature of detecting all failed tests, and then waiting
for you to modify files and continuously re-rerun those failing tests
until they pass while you fix them. So that you don't need to re start
pytest after you made the fix. This is repeated until all tests pass
after which again a full run is performed.

\begin{Shaded}
\begin{Highlighting}[]
\ExtensionTok{pip}\NormalTok{ install pytest{-}xdist}
\end{Highlighting}
\end{Shaded}

To enter the mode: \texttt{pytest\ -f} or
\texttt{pytest\ -\/-looponfail}

File changes are detected by looking at \texttt{looponfailroots} root
directories and all of their contents (recursively). If the default for
this value does not work for you, you can change it in your project by
setting a configuration option in \texttt{setup.cfg}:

\begin{Shaded}
\begin{Highlighting}[]
\KeywordTok{[tool:pytest]}
\DataTypeTok{looponfailroots }\OtherTok{=}\StringTok{ transformers tests}
\end{Highlighting}
\end{Shaded}

or \texttt{pytest.ini}/\texttt{tox.ini} files:

\begin{Shaded}
\begin{Highlighting}[]
\KeywordTok{[pytest]}
\DataTypeTok{looponfailroots }\OtherTok{=}\StringTok{ transformers tests}
\end{Highlighting}
\end{Shaded}

This would lead to only looking for file changes in the respective
directories, specified relatively to the ini-file's directory.

\href{https://github.com/joeyespo/pytest-watch}{pytest-watch} is an
alternative implementation of this functionality.

\subsection{Skip a test module}\label{skip-a-test-module}

If you want to run all test modules, except a few you can exclude them
by giving an explicit list of tests to run. For example, to run all
except \texttt{test\_modeling\_*.py} tests:

\begin{Shaded}
\begin{Highlighting}[]
\ExtensionTok{pytest} \VariableTok{$(}\FunctionTok{ls} \AttributeTok{{-}1}\NormalTok{ tests/}\PreprocessorTok{*}\NormalTok{py }\KeywordTok{|} \FunctionTok{grep} \AttributeTok{{-}v}\NormalTok{ test\_modeling}\VariableTok{)}
\end{Highlighting}
\end{Shaded}

\subsection{Clearing state}\label{clearing-state}

CI builds and when isolation is important (against speed), cache should
be cleared:

\begin{Shaded}
\begin{Highlighting}[]
\ExtensionTok{pytest} \AttributeTok{{-}{-}cache{-}clear}\NormalTok{ tests}
\end{Highlighting}
\end{Shaded}

\subsection{Running tests in parallel}\label{running-tests-in-parallel}

As mentioned earlier \texttt{make\ test} runs tests in parallel via
\texttt{pytest-xdist} plugin (\texttt{-n\ X} argument,
e.g.~\texttt{-n\ 2} to run 2 parallel jobs).

\texttt{pytest-xdist}'s \texttt{-\/-dist=} option allows one to control
how the tests are grouped. \texttt{-\/-dist=loadfile} puts the tests
located in one file onto the same process.

Since the order of executed tests is different and unpredictable, if
running the test suite with \texttt{pytest-xdist} produces failures
(meaning we have some undetected coupled tests), use
\href{https://github.com/ESSS/pytest-replay}{pytest-replay} to replay
the tests in the same order, which should help with then somehow
reducing that failing sequence to a minimum.

\subsection{Test order and repetition}\label{test-order-and-repetition}

It's good to repeat the tests several times, in sequence, randomly, or
in sets, to detect any potential inter-dependency and state-related bugs
(tear down). And the straightforward multiple repetition is just good to
detect some problems that get uncovered by randomness of DL.

\subsubsection{Repeat tests}\label{repeat-tests}

\begin{itemize}
\tightlist
\item
  \href{https://github.com/dropbox/pytest-flakefinder}{pytest-flakefinder}:
\end{itemize}

\begin{Shaded}
\begin{Highlighting}[]
\ExtensionTok{pip}\NormalTok{ install pytest{-}flakefinder}
\end{Highlighting}
\end{Shaded}

And then run every test multiple times (50 by default):

\begin{Shaded}
\begin{Highlighting}[]
\ExtensionTok{pytest} \AttributeTok{{-}{-}flake{-}finder} \AttributeTok{{-}{-}flake{-}runs}\OperatorTok{=}\NormalTok{5 tests/test\_failing\_test.py}
\end{Highlighting}
\end{Shaded}

footnote: This plugin doesn't work with \texttt{-n} flag from
\texttt{pytest-xdist}.

footnote: There is another plugin \texttt{pytest-repeat}, but it doesn't
work with \texttt{unittest}.

\subsubsection{Run tests in a random
order}\label{run-tests-in-a-random-order}

\begin{Shaded}
\begin{Highlighting}[]
\ExtensionTok{pip}\NormalTok{ install pytest{-}random{-}order}
\end{Highlighting}
\end{Shaded}

Important: the presence of \texttt{pytest-random-order} will
automatically randomize tests, no configuration change or command line
options is required.

As explained earlier this allows detection of coupled tests - where one
test's state affects the state of another. When
\texttt{pytest-random-order} is installed it will print the random seed
it used for that session, e.g:

\begin{Shaded}
\begin{Highlighting}[]
\ExtensionTok{pytest}\NormalTok{ tests}
\ExtensionTok{[...]}
\ExtensionTok{Using} \AttributeTok{{-}{-}random{-}order{-}bucket}\OperatorTok{=}\NormalTok{module}
\ExtensionTok{Using} \AttributeTok{{-}{-}random{-}order{-}seed}\OperatorTok{=}\NormalTok{573663}
\end{Highlighting}
\end{Shaded}

So that if the given particular sequence fails, you can reproduce it by
adding that exact seed, e.g.:

\begin{Shaded}
\begin{Highlighting}[]
\ExtensionTok{pytest} \AttributeTok{{-}{-}random{-}order{-}seed}\OperatorTok{=}\NormalTok{573663}
\ExtensionTok{[...]}
\ExtensionTok{Using} \AttributeTok{{-}{-}random{-}order{-}bucket}\OperatorTok{=}\NormalTok{module}
\ExtensionTok{Using} \AttributeTok{{-}{-}random{-}order{-}seed}\OperatorTok{=}\NormalTok{573663}
\end{Highlighting}
\end{Shaded}

It will only reproduce the exact order if you use the exact same list of
tests (or no list at all). Once you start to manually narrowing down the
list you can no longer rely on the seed, but have to list them manually
in the exact order they failed and tell pytest to not randomize them
instead using \texttt{-\/-random-order-bucket=none}, e.g.:

\begin{Shaded}
\begin{Highlighting}[]
\ExtensionTok{pytest} \AttributeTok{{-}{-}random{-}order{-}bucket}\OperatorTok{=}\NormalTok{none tests/test\_a.py tests/test\_c.py tests/test\_b.py}
\end{Highlighting}
\end{Shaded}

To disable the shuffling for all tests:

\begin{Shaded}
\begin{Highlighting}[]
\ExtensionTok{pytest} \AttributeTok{{-}{-}random{-}order{-}bucket}\OperatorTok{=}\NormalTok{none}
\end{Highlighting}
\end{Shaded}

By default \texttt{-\/-random-order-bucket=module} is implied, which
will shuffle the files on the module levels. It can also shuffle on
\texttt{class}, \texttt{package}, \texttt{global} and \texttt{none}
levels. For the complete details please see its
\href{https://github.com/jbasko/pytest-random-order}{documentation}.

Another randomization alternative is:
\href{https://github.com/pytest-dev/pytest-randomly}{\texttt{pytest-randomly}}.
This module has a very similar functionality/interface, but it doesn't
have the bucket modes available in \texttt{pytest-random-order}. It has
the same problem of imposing itself once installed.

\subsection{Look and feel variations}\label{look-and-feel-variations}

\subsubsection{pytest-sugar}\label{pytest-sugar}

\href{https://github.com/Frozenball/pytest-sugar}{pytest-sugar} is a
plugin that improves the look-n-feel, adds a progressbar, and show tests
that fail and the assert instantly. It gets activated automatically upon
installation.

\begin{Shaded}
\begin{Highlighting}[]
\ExtensionTok{pip}\NormalTok{ install pytest{-}sugar}
\end{Highlighting}
\end{Shaded}

To run tests without it, run:

\begin{Shaded}
\begin{Highlighting}[]
\ExtensionTok{pytest} \AttributeTok{{-}p}\NormalTok{ no:sugar}
\end{Highlighting}
\end{Shaded}

or uninstall it.

\subsubsection{Report each sub-test name and its
progress}\label{report-each-sub-test-name-and-its-progress}

For a single or a group of tests via \texttt{pytest} (after
\texttt{pip\ install\ pytest-pspec}):

\begin{Shaded}
\begin{Highlighting}[]
\ExtensionTok{pytest} \AttributeTok{{-}{-}pspec}\NormalTok{ tests/test\_optimization.py}
\end{Highlighting}
\end{Shaded}

\subsubsection{Instantly shows failed
tests}\label{instantly-shows-failed-tests}

\href{https://github.com/pytest-dev/pytest-instafail}{pytest-instafail}
shows failures and errors instantly instead of waiting until the end of
test session.

\begin{Shaded}
\begin{Highlighting}[]
\ExtensionTok{pip}\NormalTok{ install pytest{-}instafail}
\end{Highlighting}
\end{Shaded}

\begin{Shaded}
\begin{Highlighting}[]
\ExtensionTok{pytest} \AttributeTok{{-}{-}instafail}
\end{Highlighting}
\end{Shaded}

\subsection{To GPU or not to GPU}\label{to-gpu-or-not-to-gpu}

On a GPU-enabled setup, to test in CPU-only mode add
\texttt{CUDA\_VISIBLE\_DEVICES=""}:

\begin{Shaded}
\begin{Highlighting}[]
\VariableTok{CUDA\_VISIBLE\_DEVICES}\OperatorTok{=}\StringTok{""} \ExtensionTok{pytest}\NormalTok{ tests/utils/test\_logging.py}
\end{Highlighting}
\end{Shaded}

or if you have multiple gpus, you can specify which one is to be used by
\texttt{pytest}. For example, to use only the second gpu if you have
gpus \texttt{0} and \texttt{1}, you can run:

\begin{Shaded}
\begin{Highlighting}[]
\VariableTok{CUDA\_VISIBLE\_DEVICES}\OperatorTok{=}\StringTok{"1"} \ExtensionTok{pytest}\NormalTok{ tests/utils/test\_logging.py}
\end{Highlighting}
\end{Shaded}

This is handy when you want to run different tasks on different GPUs.

Some tests must be run on CPU-only, others on either CPU or GPU or TPU,
yet others on multiple-GPUs. The following skip decorators are used to
set the requirements of tests CPU/GPU/TPU-wise:

\begin{itemize}
\tightlist
\item
  \texttt{require\_torch} - this test will run only under torch
\item
  \texttt{require\_torch\_gpu} - as \texttt{require\_torch} plus
  requires at least 1 GPU
\item
  \texttt{require\_torch\_multi\_gpu} - as \texttt{require\_torch} plus
  requires at least 2 GPUs
\item
  \texttt{require\_torch\_non\_multi\_gpu} - as \texttt{require\_torch}
  plus requires 0 or 1 GPUs
\item
  \texttt{require\_torch\_up\_to\_2\_gpus} - as \texttt{require\_torch}
  plus requires 0 or 1 or 2 GPUs
\item
  \texttt{require\_torch\_tpu} - as \texttt{require\_torch} plus
  requires at least 1 TPU
\end{itemize}

Let's depict the GPU requirements in the following table:

\begin{longtable}[]{@{}ll@{}}
\toprule\noalign{}
n gpus & decorator \\
\midrule\noalign{}
\endhead
\bottomrule\noalign{}
\endlastfoot
\texttt{\textgreater{}=\ 0} & \texttt{@require\_torch} \\
\texttt{\textgreater{}=\ 1} & \texttt{@require\_torch\_gpu} \\
\texttt{\textgreater{}=\ 2} & \texttt{@require\_torch\_multi\_gpu} \\
\texttt{\textless{}\ 2} & \texttt{@require\_torch\_non\_multi\_gpu} \\
\texttt{\textless{}\ 3} & \texttt{@require\_torch\_up\_to\_2\_gpus} \\
\end{longtable}

For example, here is a test that must be run only when there are 2 or
more GPUs available and pytorch is installed:

\begin{Shaded}
\begin{Highlighting}[]
\ImportTok{from}\NormalTok{ testing\_utils }\ImportTok{import}\NormalTok{ require\_torch\_multi\_gpu}

\AttributeTok{@require\_torch\_multi\_gpu}
\KeywordTok{def}\NormalTok{ test\_example\_with\_multi\_gpu():}
\end{Highlighting}
\end{Shaded}

These decorators can be stacked:

\begin{Shaded}
\begin{Highlighting}[]
\ImportTok{from}\NormalTok{ testing\_utils }\ImportTok{import}\NormalTok{ require\_torch\_gpu}

\AttributeTok{@require\_torch\_gpu}
\AttributeTok{@some\_other\_decorator}
\KeywordTok{def}\NormalTok{ test\_example\_slow\_on\_gpu():}
\end{Highlighting}
\end{Shaded}

Some decorators like \texttt{@parametrized} rewrite test names,
therefore \texttt{@require\_*} skip decorators have to be listed last
for them to work correctly. Here is an example of the correct usage:

\begin{Shaded}
\begin{Highlighting}[]
\ImportTok{from}\NormalTok{ testing\_utils }\ImportTok{import}\NormalTok{ require\_torch\_multi\_gpu}
\ImportTok{from}\NormalTok{ parameterized }\ImportTok{import}\NormalTok{ parameterized}

\AttributeTok{@parameterized.expand}\NormalTok{(...)}
\AttributeTok{@require\_torch\_multi\_gpu}
\KeywordTok{def}\NormalTok{ test\_integration\_foo():}
\end{Highlighting}
\end{Shaded}

This order problem doesn't exist with \texttt{@pytest.mark.parametrize},
you can put it first or last and it will still work. But it only works
with non-unittests.

Inside tests:

\begin{itemize}
\item
  How many GPUs are available:

\begin{Shaded}
\begin{Highlighting}[]
\ImportTok{from}\NormalTok{ testing\_utils }\ImportTok{import}\NormalTok{ get\_gpu\_count}

\NormalTok{n\_gpu }\OperatorTok{=}\NormalTok{ get\_gpu\_count()}
\end{Highlighting}
\end{Shaded}
\end{itemize}

\subsection{Distributed training}\label{distributed-training}

\texttt{pytest} can't deal with distributed training directly. If this
is attempted - the sub-processes don't do the right thing and end up
thinking they are \texttt{pytest} and start running the test suite in
loops. It works, however, if one spawns a normal process that then
spawns off multiple workers and manages the IO pipes.

Here are some tests that use it:

\begin{itemize}
\tightlist
\item
  \href{https://github.com/huggingface/transformers/blob/58e3d23e97078f361a533b9ec4a6a2de674ea52a/tests/trainer/test_trainer_distributed.py}{test\_trainer\_distributed.py}
\item
  \href{https://github.com/huggingface/transformers/blob/58e3d23e97078f361a533b9ec4a6a2de674ea52a/tests/deepspeed/test_deepspeed.py}{test\_deepspeed.py}
\end{itemize}

To jump right into the execution point, search for the
\texttt{execute\_subprocess\_async} call in those tests, which you will
find inside \url{testing_utils.py}.

You will need at least 2 GPUs to see these tests in action:

\begin{Shaded}
\begin{Highlighting}[]
\VariableTok{CUDA\_VISIBLE\_DEVICES}\OperatorTok{=}\NormalTok{0,1 }\VariableTok{RUN\_SLOW}\OperatorTok{=}\NormalTok{1 }\ExtensionTok{pytest} \AttributeTok{{-}sv}\NormalTok{ tests/test\_trainer\_distributed.py}
\end{Highlighting}
\end{Shaded}

(\texttt{RUN\_SLOW} is a special decorator used by HF Transformers to
normally skip heavy tests)

\subsection{Output capture}\label{output-capture}

During test execution any output sent to \texttt{stdout} and
\texttt{stderr} is captured. If a test or a setup method fails, its
according captured output will usually be shown along with the failure
traceback.

To disable output capturing and to get the \texttt{stdout} and
\texttt{stderr} normally, use \texttt{-s} or \texttt{-\/-capture=no}:

\begin{Shaded}
\begin{Highlighting}[]
\ExtensionTok{pytest} \AttributeTok{{-}s}\NormalTok{ tests/utils/test\_logging.py}
\end{Highlighting}
\end{Shaded}

To send test results to JUnit format output:

\begin{Shaded}
\begin{Highlighting}[]
\ExtensionTok{py.test}\NormalTok{ tests }\AttributeTok{{-}{-}junitxml}\OperatorTok{=}\NormalTok{result.xml}
\end{Highlighting}
\end{Shaded}

\subsection{Color control}\label{color-control}

To have no color (e.g., yellow on white background is not readable):

\begin{Shaded}
\begin{Highlighting}[]
\ExtensionTok{pytest} \AttributeTok{{-}{-}color}\OperatorTok{=}\NormalTok{no tests/utils/test\_logging.py}
\end{Highlighting}
\end{Shaded}

\subsection{Sending test report to online pastebin
service}\label{sending-test-report-to-online-pastebin-service}

Creating a URL for each test failure:

\begin{Shaded}
\begin{Highlighting}[]
\ExtensionTok{pytest} \AttributeTok{{-}{-}pastebin}\OperatorTok{=}\NormalTok{failed tests/utils/test\_logging.py}
\end{Highlighting}
\end{Shaded}

This will submit test run information to a remote Paste service and
provide a URL for each failure. You may select tests as usual or add for
example -x if you only want to send one particular failure.

Creating a URL for a whole test session log:

\begin{Shaded}
\begin{Highlighting}[]
\ExtensionTok{pytest} \AttributeTok{{-}{-}pastebin}\OperatorTok{=}\NormalTok{all tests/utils/test\_logging.py}
\end{Highlighting}
\end{Shaded}

\section{Writing tests}\label{writing-tests}

Most of the time if combining \texttt{pytest} and \texttt{unittest} in
the same test suite works just fine. You can read
\href{https://docs.pytest.org/en/stable/unittest.html}{here} which
features are supported when doing that , but the important thing to
remember is that most \texttt{pytest} fixtures don't work. Neither
parametrization, but we use the module \texttt{parameterized} that works
in a similar way.

\subsection{Parametrization}\label{parametrization}

Often, there is a need to run the same test multiple times, but with
different arguments. It could be done from within the test, but then
there is no way of running that test for just one set of arguments.

\begin{Shaded}
\begin{Highlighting}[]
\CommentTok{\# test\_this1.py}
\ImportTok{import}\NormalTok{ unittest}
\ImportTok{from}\NormalTok{ parameterized }\ImportTok{import}\NormalTok{ parameterized}


\KeywordTok{class}\NormalTok{ TestMathUnitTest(unittest.TestCase):}
    \AttributeTok{@parameterized.expand}\NormalTok{(}
\NormalTok{        [}
\NormalTok{            (}\StringTok{"negative"}\NormalTok{, }\OperatorTok{{-}}\FloatTok{1.5}\NormalTok{, }\OperatorTok{{-}}\FloatTok{2.0}\NormalTok{),}
\NormalTok{            (}\StringTok{"integer"}\NormalTok{, }\DecValTok{1}\NormalTok{, }\FloatTok{1.0}\NormalTok{),}
\NormalTok{            (}\StringTok{"large fraction"}\NormalTok{, }\FloatTok{1.6}\NormalTok{, }\DecValTok{1}\NormalTok{),}
\NormalTok{        ]}
\NormalTok{    )}
    \KeywordTok{def}\NormalTok{ test\_floor(}\VariableTok{self}\NormalTok{, name, }\BuiltInTok{input}\NormalTok{, expected):}
\NormalTok{        assert\_equal(math.floor(}\BuiltInTok{input}\NormalTok{), expected)}
\end{Highlighting}
\end{Shaded}

Now, by default this test will be run 3 times, each time with the last 3
arguments of \texttt{test\_floor} being assigned the corresponding
arguments in the parameter list.

And you could run just the \texttt{negative} and \texttt{integer} sets
of params with:

\begin{Shaded}
\begin{Highlighting}[]
\ExtensionTok{pytest} \AttributeTok{{-}k} \StringTok{"negative and integer"}\NormalTok{ tests/test\_mytest.py}
\end{Highlighting}
\end{Shaded}

or all but \texttt{negative} sub-tests, with:

\begin{Shaded}
\begin{Highlighting}[]
\ExtensionTok{pytest} \AttributeTok{{-}k} \StringTok{"not negative"}\NormalTok{ tests/test\_mytest.py}
\end{Highlighting}
\end{Shaded}

Besides using the \texttt{-k} filter that was just mentioned, you can
find out the exact name of each sub-test and run any or all of them
using their exact names.

\begin{Shaded}
\begin{Highlighting}[]
\ExtensionTok{pytest}\NormalTok{ test\_this1.py }\AttributeTok{{-}{-}collect{-}only} \AttributeTok{{-}q}
\end{Highlighting}
\end{Shaded}

and it will list:

\begin{Shaded}
\begin{Highlighting}[]
\ExtensionTok{test\_this1.py::TestMathUnitTest::test\_floor\_0\_negative}
\ExtensionTok{test\_this1.py::TestMathUnitTest::test\_floor\_1\_integer}
\ExtensionTok{test\_this1.py::TestMathUnitTest::test\_floor\_2\_large\_fraction}
\end{Highlighting}
\end{Shaded}

So now you can run just 2 specific sub-tests:

\begin{Shaded}
\begin{Highlighting}[]
\ExtensionTok{pytest}\NormalTok{ test\_this1.py::TestMathUnitTest::test\_floor\_0\_negative  test\_this1.py::TestMathUnitTest::test\_floor\_1\_integer}
\end{Highlighting}
\end{Shaded}

The module \href{https://pypi.org/project/parameterized/}{parameterized}
works for both: \texttt{unittests} and \texttt{pytest} tests.

If, however, the test is not a \texttt{unittest}, you may use
\texttt{pytest.mark.parametrize}.

Here is the same example, this time using \texttt{pytest}'s
\texttt{parametrize} marker:

\begin{Shaded}
\begin{Highlighting}[]
\CommentTok{\# test\_this2.py}
\ImportTok{import}\NormalTok{ pytest}


\AttributeTok{@pytest.mark.parametrize}\NormalTok{(}
    \StringTok{"name, input, expected"}\NormalTok{,}
\NormalTok{    [}
\NormalTok{        (}\StringTok{"negative"}\NormalTok{, }\OperatorTok{{-}}\FloatTok{1.5}\NormalTok{, }\OperatorTok{{-}}\FloatTok{2.0}\NormalTok{),}
\NormalTok{        (}\StringTok{"integer"}\NormalTok{, }\DecValTok{1}\NormalTok{, }\FloatTok{1.0}\NormalTok{),}
\NormalTok{        (}\StringTok{"large fraction"}\NormalTok{, }\FloatTok{1.6}\NormalTok{, }\DecValTok{1}\NormalTok{),}
\NormalTok{    ],}
\NormalTok{)}
\KeywordTok{def}\NormalTok{ test\_floor(name, }\BuiltInTok{input}\NormalTok{, expected):}
\NormalTok{    assert\_equal(math.floor(}\BuiltInTok{input}\NormalTok{), expected)}
\end{Highlighting}
\end{Shaded}

Same as with \texttt{parameterized}, with
\texttt{pytest.mark.parametrize} you can have a fine control over which
sub-tests are run, if the \texttt{-k} filter doesn't do the job. Except,
this parametrization function creates a slightly different set of names
for the sub-tests. Here is what they look like:

\begin{Shaded}
\begin{Highlighting}[]
\ExtensionTok{pytest}\NormalTok{ test\_this2.py }\AttributeTok{{-}{-}collect{-}only} \AttributeTok{{-}q}
\end{Highlighting}
\end{Shaded}

and it will list:

\begin{Shaded}
\begin{Highlighting}[]
\ExtensionTok{test\_this2.py::test\_floor[integer{-}1{-}1.0]}
\ExtensionTok{test\_this2.py::test\_floor[negative{-}{-}1.5{-}{-}2.0]}
\ExtensionTok{test\_this2.py::test\_floor[large}\NormalTok{ fraction{-}1.6{-}1]}
\end{Highlighting}
\end{Shaded}

So now you can run just the specific test:

\begin{Shaded}
\begin{Highlighting}[]
\ExtensionTok{pytest}\NormalTok{ test\_this2.py::test\_floor}\PreprocessorTok{[}\SpecialStringTok{negative}\PreprocessorTok{{-}{-}}\SpecialStringTok{1.5}\PreprocessorTok{{-}{-}}\SpecialStringTok{2.0}\PreprocessorTok{]}\NormalTok{ test\_this2.py::test\_floor}\PreprocessorTok{[}\SpecialStringTok{integer}\PreprocessorTok{{-}}\SpecialStringTok{1}\PreprocessorTok{{-}}\SpecialStringTok{1.0}\PreprocessorTok{]}
\end{Highlighting}
\end{Shaded}

as in the previous example.

\subsection{Files and directories}\label{files-and-directories}

In tests often we need to know where things are relative to the current
test file, and it's not trivial since the test could be invoked from
more than one directory or could reside in sub-directories with
different depths. A helper class \texttt{testing\_utils.TestCasePlus}
solves this problem by sorting out all the basic paths and provides easy
accessors to them:

\begin{itemize}
\item
  \texttt{pathlib} objects (all fully resolved):

  \begin{itemize}
  \tightlist
  \item
    \texttt{test\_file\_path} - the current test file path,
    i.e.~\texttt{\_\_file\_\_}
  \item
    \texttt{test\_file\_dir} - the directory containing the current test
    file
  \item
    \texttt{tests\_dir} - the directory of the \texttt{tests} test suite
  \item
    \texttt{examples\_dir} - the directory of the \texttt{examples} test
    suite
  \item
    \texttt{repo\_root\_dir} - the directory of the repository
  \item
    \texttt{src\_dir} - the directory of \texttt{src} (i.e.~where the
    \texttt{transformers} sub-dir resides)
  \end{itemize}
\item
  stringified paths -- same as above but these return paths as strings,
  rather than \texttt{pathlib} objects:

  \begin{itemize}
  \tightlist
  \item
    \texttt{test\_file\_path\_str}
  \item
    \texttt{test\_file\_dir\_str}
  \item
    \texttt{tests\_dir\_str}
  \item
    \texttt{examples\_dir\_str}
  \item
    \texttt{repo\_root\_dir\_str}
  \item
    \texttt{src\_dir\_str}
  \end{itemize}
\end{itemize}

To start using those all you need is to make sure that the test resides
in a subclass of \texttt{testing\_utils.TestCasePlus}. For example:

\begin{Shaded}
\begin{Highlighting}[]
\ImportTok{from}\NormalTok{ testing\_utils }\ImportTok{import}\NormalTok{ TestCasePlus}


\KeywordTok{class}\NormalTok{ PathExampleTest(TestCasePlus):}
    \KeywordTok{def}\NormalTok{ test\_something\_involving\_local\_locations(}\VariableTok{self}\NormalTok{):}
\NormalTok{        data\_dir }\OperatorTok{=} \VariableTok{self}\NormalTok{.tests\_dir }\OperatorTok{/} \StringTok{"fixtures/tests\_samples/wmt\_en\_ro"}
\end{Highlighting}
\end{Shaded}

If you don't need to manipulate paths via \texttt{pathlib} or you just
need a path as a string, you can always invoked \texttt{str()} on the
\texttt{pathlib} object or use the accessors ending with \texttt{\_str}.
For example:

\begin{Shaded}
\begin{Highlighting}[]
\ImportTok{from}\NormalTok{ testing\_utils }\ImportTok{import}\NormalTok{ TestCasePlus}


\KeywordTok{class}\NormalTok{ PathExampleTest(TestCasePlus):}
    \KeywordTok{def}\NormalTok{ test\_something\_involving\_stringified\_locations(}\VariableTok{self}\NormalTok{):}
\NormalTok{        examples\_dir }\OperatorTok{=} \VariableTok{self}\NormalTok{.examples\_dir\_str}
\end{Highlighting}
\end{Shaded}

\subsubsection{Temporary files and
directories}\label{temporary-files-and-directories}

Using unique temporary files and directories are essential for parallel
test running, so that the tests won't overwrite each other's data. Also
we want to get the temporary files and directories removed at the end of
each test that created them. Therefore, using packages like
\texttt{tempfile}, which address these needs is essential.

However, when debugging tests, you need to be able to see what goes into
the temporary file or directory and you want to know it's exact path and
not having it randomized on every test re-run.

A helper class \texttt{testing\_utils.TestCasePlus} is best used for
such purposes. It's a sub-class of \texttt{unittest.TestCase}, so we can
easily inherit from it in the test modules.

Here is an example of its usage:

\begin{Shaded}
\begin{Highlighting}[]
\ImportTok{from}\NormalTok{ testing\_utils }\ImportTok{import}\NormalTok{ TestCasePlus}


\KeywordTok{class}\NormalTok{ ExamplesTests(TestCasePlus):}
    \KeywordTok{def}\NormalTok{ test\_whatever(}\VariableTok{self}\NormalTok{):}
\NormalTok{        tmp\_dir }\OperatorTok{=} \VariableTok{self}\NormalTok{.get\_auto\_remove\_tmp\_dir()}
\end{Highlighting}
\end{Shaded}

This code creates a unique temporary directory, and sets
\texttt{tmp\_dir} to its location.

\begin{itemize}
\tightlist
\item
  Create a unique temporary dir:
\end{itemize}

\begin{Shaded}
\begin{Highlighting}[]
\KeywordTok{def}\NormalTok{ test\_whatever(}\VariableTok{self}\NormalTok{):}
\NormalTok{    tmp\_dir }\OperatorTok{=} \VariableTok{self}\NormalTok{.get\_auto\_remove\_tmp\_dir()}
\end{Highlighting}
\end{Shaded}

\texttt{tmp\_dir} will contain the path to the created temporary dir. It
will be automatically removed at the end of the test.

\begin{itemize}
\tightlist
\item
  Create a temporary dir of my choice, ensure it's empty before the test
  starts and don't empty it after the test.
\end{itemize}

\begin{Shaded}
\begin{Highlighting}[]
\KeywordTok{def}\NormalTok{ test\_whatever(}\VariableTok{self}\NormalTok{):}
\NormalTok{    tmp\_dir }\OperatorTok{=} \VariableTok{self}\NormalTok{.get\_auto\_remove\_tmp\_dir(}\StringTok{"./xxx"}\NormalTok{)}
\end{Highlighting}
\end{Shaded}

This is useful for debug when you want to monitor a specific directory
and want to make sure the previous tests didn't leave any data in there.

\begin{itemize}
\item
  You can override the default behavior by directly overriding the
  \texttt{before} and \texttt{after} args, leading to one of the
  following behaviors:

  \begin{itemize}
  \tightlist
  \item
    \texttt{before=True}: the temporary dir will always be cleared at
    the beginning of the test.
  \item
    \texttt{before=False}: if the temporary dir already existed, any
    existing files will remain there.
  \item
    \texttt{after=True}: the temporary dir will always be deleted at the
    end of the test.
  \item
    \texttt{after=False}: the temporary dir will always be left intact
    at the end of the test.
  \end{itemize}
\end{itemize}

footnote: In order to run the equivalent of \texttt{rm\ -r} safely, only
subdirs of the project repository checkout are allowed if an explicit
\texttt{tmp\_dir} is used, so that by mistake no \texttt{/tmp} or
similar important part of the filesystem will get nuked. i.e.~please
always pass paths that start with \texttt{./}.

footnote: Each test can register multiple temporary directories and they
all will get auto-removed, unless requested otherwise.

\subsubsection{Temporary sys.path
override}\label{temporary-sys.path-override}

If you need to temporary override \texttt{sys.path} to import from
another test for example, you can use the \texttt{ExtendSysPath} context
manager. Example:

\begin{Shaded}
\begin{Highlighting}[]
\ImportTok{import}\NormalTok{ os}
\ImportTok{from}\NormalTok{ testing\_utils }\ImportTok{import}\NormalTok{ ExtendSysPath}

\NormalTok{bindir }\OperatorTok{=}\NormalTok{ os.path.abspath(os.path.dirname(}\VariableTok{\_\_file\_\_}\NormalTok{))}
\ControlFlowTok{with}\NormalTok{ ExtendSysPath(}\SpecialStringTok{f"}\SpecialCharTok{\{}\NormalTok{bindir}\SpecialCharTok{\}}\SpecialStringTok{/.."}\NormalTok{):}
    \ImportTok{from}\NormalTok{ test\_trainer }\ImportTok{import}\NormalTok{ TrainerIntegrationCommon  }\CommentTok{\# noqa}
\end{Highlighting}
\end{Shaded}

\subsection{Skipping tests}\label{skipping-tests}

This is useful when a bug is found and a new test is written, yet the
bug is not fixed yet. In order to be able to commit it to the main
repository we need make sure it's skipped during \texttt{make\ test}.

Methods:

\begin{itemize}
\item
  A \textbf{skip} means that you expect your test to pass only if some
  conditions are met, otherwise pytest should skip running the test
  altogether. Common examples are skipping windows-only tests on
  non-windows platforms, or skipping tests that depend on an external
  resource which is not available at the moment (for example a
  database).
\item
  A \textbf{xfail} means that you expect a test to fail for some reason.
  A common example is a test for a feature not yet implemented, or a bug
  not yet fixed. When a test passes despite being expected to fail
  (marked with \texttt{pytest.mark.xfail}), it's an xpass and will be
  reported in the test summary.
\end{itemize}

One of the important differences between the two is that \texttt{skip}
doesn't run the test, and \texttt{xfail} does. So if the code that's
buggy causes some bad state that will affect other tests, do not use
\texttt{xfail}.

\subsubsection{Implementation}\label{implementation}

\begin{itemize}
\item
  Here is how to skip whole test unconditionally:

\begin{Shaded}
\begin{Highlighting}[]
\AttributeTok{@unittest.skip}\NormalTok{(}\StringTok{"this bug needs to be fixed"}\NormalTok{)}
\KeywordTok{def}\NormalTok{ test\_feature\_x():}
\end{Highlighting}
\end{Shaded}

  or via pytest:

\begin{Shaded}
\begin{Highlighting}[]
\AttributeTok{@pytest.mark.skip}\NormalTok{(reason}\OperatorTok{=}\StringTok{"this bug needs to be fixed"}\NormalTok{)}
\end{Highlighting}
\end{Shaded}

  or the \texttt{xfail} way:

\begin{Shaded}
\begin{Highlighting}[]
\AttributeTok{@pytest.mark.xfail}
\KeywordTok{def}\NormalTok{ test\_feature\_x():}
\end{Highlighting}
\end{Shaded}
\item
  Here's how to skip a test based on internal checks within the test:

\begin{Shaded}
\begin{Highlighting}[]
\KeywordTok{def}\NormalTok{ test\_feature\_x():}
    \ControlFlowTok{if} \KeywordTok{not}\NormalTok{ has\_something():}
\NormalTok{        pytest.skip(}\StringTok{"unsupported configuration"}\NormalTok{)}
\end{Highlighting}
\end{Shaded}

  or the whole module:

\begin{Shaded}
\begin{Highlighting}[]
\ImportTok{import}\NormalTok{ pytest}

\ControlFlowTok{if} \KeywordTok{not}\NormalTok{ pytest.config.getoption(}\StringTok{"{-}{-}custom{-}flag"}\NormalTok{):}
\NormalTok{    pytest.skip(}\StringTok{"{-}{-}custom{-}flag is missing, skipping tests"}\NormalTok{, allow\_module\_level}\OperatorTok{=}\VariableTok{True}\NormalTok{)}
\end{Highlighting}
\end{Shaded}

  or the \texttt{xfail} way:

\begin{Shaded}
\begin{Highlighting}[]
\KeywordTok{def}\NormalTok{ test\_feature\_x():}
\NormalTok{    pytest.xfail(}\StringTok{"expected to fail until bug XYZ is fixed"}\NormalTok{)}
\end{Highlighting}
\end{Shaded}
\item
  Here is how to skip all tests in a module if some import is missing:

\begin{Shaded}
\begin{Highlighting}[]
\NormalTok{docutils }\OperatorTok{=}\NormalTok{ pytest.importorskip(}\StringTok{"docutils"}\NormalTok{, minversion}\OperatorTok{=}\StringTok{"0.3"}\NormalTok{)}
\end{Highlighting}
\end{Shaded}
\item
  Skip a test based on a condition:

\begin{Shaded}
\begin{Highlighting}[]
\AttributeTok{@pytest.mark.skipif}\NormalTok{(sys.version\_info }\OperatorTok{\textless{}}\NormalTok{ (}\DecValTok{3}\NormalTok{,}\DecValTok{6}\NormalTok{), reason}\OperatorTok{=}\StringTok{"requires python3.6 or higher"}\NormalTok{)}
\KeywordTok{def}\NormalTok{ test\_feature\_x():}
\end{Highlighting}
\end{Shaded}

  or:

\begin{Shaded}
\begin{Highlighting}[]
\AttributeTok{@unittest.skipIf}\NormalTok{(torch\_device }\OperatorTok{==} \StringTok{"cpu"}\NormalTok{, }\StringTok{"Can\textquotesingle{}t do half precision"}\NormalTok{)}
\KeywordTok{def}\NormalTok{ test\_feature\_x():}
\end{Highlighting}
\end{Shaded}

  or skip the whole module:

\begin{Shaded}
\begin{Highlighting}[]
\AttributeTok{@pytest.mark.skipif}\NormalTok{(sys.platform }\OperatorTok{==} \StringTok{\textquotesingle{}win32\textquotesingle{}}\NormalTok{, reason}\OperatorTok{=}\StringTok{"does not run on windows"}\NormalTok{)}
\KeywordTok{class}\NormalTok{ TestClass():}
    \KeywordTok{def}\NormalTok{ test\_feature\_x(}\VariableTok{self}\NormalTok{):}
\end{Highlighting}
\end{Shaded}
\end{itemize}

More details, example and ways are
\href{https://docs.pytest.org/en/latest/skipping.html}{here}.

\subsection{Capturing outputs}\label{capturing-outputs}

\subsubsection{Capturing the stdout/stderr
output}\label{capturing-the-stdoutstderr-output}

In order to test functions that write to \texttt{stdout} and/or
\texttt{stderr}, the test can access those streams using the
\texttt{pytest}'s
\href{https://docs.pytest.org/en/latest/capture.html}{capsys system}.
Here is how this is accomplished:

\begin{Shaded}
\begin{Highlighting}[]
\ImportTok{import}\NormalTok{ sys}

\KeywordTok{def}\NormalTok{ print\_to\_stdout(s):}
    \BuiltInTok{print}\NormalTok{(s)}


\KeywordTok{def}\NormalTok{ print\_to\_stderr(s):}
\NormalTok{    sys.stderr.write(s)}


\KeywordTok{def}\NormalTok{ test\_result\_and\_stdout(capsys):}
\NormalTok{    msg }\OperatorTok{=} \StringTok{"Hello"}
\NormalTok{    print\_to\_stdout(msg)}
\NormalTok{    print\_to\_stderr(msg)}
\NormalTok{    out, err }\OperatorTok{=}\NormalTok{ capsys.readouterr()  }\CommentTok{\# consume the captured output streams}
    \CommentTok{\# optional: if you want to replay the consumed streams:}
\NormalTok{    sys.stdout.write(out)}
\NormalTok{    sys.stderr.write(err)}
    \CommentTok{\# test:}
    \ControlFlowTok{assert}\NormalTok{ msg }\KeywordTok{in}\NormalTok{ out}
    \ControlFlowTok{assert}\NormalTok{ msg }\KeywordTok{in}\NormalTok{ err}
\end{Highlighting}
\end{Shaded}

And, of course, most of the time, \texttt{stderr} will come as a part of
an exception, so try/except has to be used in such a case:

\begin{Shaded}
\begin{Highlighting}[]
\KeywordTok{def}\NormalTok{ raise\_exception(msg):}
    \ControlFlowTok{raise} \PreprocessorTok{ValueError}\NormalTok{(msg)}


\KeywordTok{def}\NormalTok{ test\_something\_exception():}
\NormalTok{    msg }\OperatorTok{=} \StringTok{"Not a good value"}
\NormalTok{    error }\OperatorTok{=} \StringTok{""}
    \ControlFlowTok{try}\NormalTok{:}
\NormalTok{        raise\_exception(msg)}
    \ControlFlowTok{except} \PreprocessorTok{Exception} \ImportTok{as}\NormalTok{ e:}
\NormalTok{        error }\OperatorTok{=} \BuiltInTok{str}\NormalTok{(e)}
        \ControlFlowTok{assert}\NormalTok{ msg }\KeywordTok{in}\NormalTok{ error, }\SpecialStringTok{f"}\SpecialCharTok{\{}\NormalTok{msg}\SpecialCharTok{\}}\SpecialStringTok{ is in the exception:}\CharTok{\textbackslash{}n}\SpecialCharTok{\{}\NormalTok{error}\SpecialCharTok{\}}\SpecialStringTok{"}
\end{Highlighting}
\end{Shaded}

Another approach to capturing stdout is via
\texttt{contextlib.redirect\_stdout}:

\begin{Shaded}
\begin{Highlighting}[]
\ImportTok{from}\NormalTok{ io }\ImportTok{import}\NormalTok{ StringIO}
\ImportTok{from}\NormalTok{ contextlib }\ImportTok{import}\NormalTok{ redirect\_stdout}


\KeywordTok{def}\NormalTok{ print\_to\_stdout(s):}
    \BuiltInTok{print}\NormalTok{(s)}


\KeywordTok{def}\NormalTok{ test\_result\_and\_stdout():}
\NormalTok{    msg }\OperatorTok{=} \StringTok{"Hello"}
    \BuiltInTok{buffer} \OperatorTok{=}\NormalTok{ StringIO()}
    \ControlFlowTok{with}\NormalTok{ redirect\_stdout(}\BuiltInTok{buffer}\NormalTok{):}
\NormalTok{        print\_to\_stdout(msg)}
\NormalTok{    out }\OperatorTok{=} \BuiltInTok{buffer}\NormalTok{.getvalue()}
    \CommentTok{\# optional: if you want to replay the consumed streams:}
\NormalTok{    sys.stdout.write(out)}
    \CommentTok{\# test:}
    \ControlFlowTok{assert}\NormalTok{ msg }\KeywordTok{in}\NormalTok{ out}
\end{Highlighting}
\end{Shaded}

An important potential issue with capturing stdout is that it may
contain \texttt{\textbackslash{}r} characters that in normal
\texttt{print} reset everything that has been printed so far. There is
no problem with \texttt{pytest}, but with \texttt{pytest\ -s} these
characters get included in the buffer, so to be able to have the test
run with and without \texttt{-s}, you have to make an extra cleanup to
the captured output, using
\texttt{re.sub(r\textquotesingle{}\textasciitilde{}.*\textbackslash{}r\textquotesingle{},\ \textquotesingle{}\textquotesingle{},\ buf,\ 0,\ re.M)}.

But, then we have a helper context manager wrapper to automatically take
care of it all, regardless of whether it has some
\texttt{\textbackslash{}r}'s in it or not, so it's a simple:

\begin{Shaded}
\begin{Highlighting}[]
\ImportTok{from}\NormalTok{ testing\_utils }\ImportTok{import}\NormalTok{ CaptureStdout}

\ControlFlowTok{with}\NormalTok{ CaptureStdout() }\ImportTok{as}\NormalTok{ cs:}
\NormalTok{    function\_that\_writes\_to\_stdout()}
\BuiltInTok{print}\NormalTok{(cs.out)}
\end{Highlighting}
\end{Shaded}

Here is a full test example:

\begin{Shaded}
\begin{Highlighting}[]
\ImportTok{from}\NormalTok{ testing\_utils }\ImportTok{import}\NormalTok{ CaptureStdout}

\NormalTok{msg }\OperatorTok{=} \StringTok{"Secret message}\CharTok{\textbackslash{}r}\StringTok{"}
\NormalTok{final }\OperatorTok{=} \StringTok{"Hello World"}
\ControlFlowTok{with}\NormalTok{ CaptureStdout() }\ImportTok{as}\NormalTok{ cs:}
    \BuiltInTok{print}\NormalTok{(msg }\OperatorTok{+}\NormalTok{ final)}
\ControlFlowTok{assert}\NormalTok{ cs.out }\OperatorTok{==}\NormalTok{ final }\OperatorTok{+} \StringTok{"}\CharTok{\textbackslash{}n}\StringTok{"}\NormalTok{, }\SpecialStringTok{f"captured: }\SpecialCharTok{\{}\NormalTok{cs}\SpecialCharTok{.}\NormalTok{out}\SpecialCharTok{\}}\SpecialStringTok{, expecting }\SpecialCharTok{\{}\NormalTok{final}\SpecialCharTok{\}}\SpecialStringTok{"}
\end{Highlighting}
\end{Shaded}

If you'd like to capture \texttt{stderr} use the \texttt{CaptureStderr}
class instead:

\begin{Shaded}
\begin{Highlighting}[]
\ImportTok{from}\NormalTok{ testing\_utils }\ImportTok{import}\NormalTok{ CaptureStderr}

\ControlFlowTok{with}\NormalTok{ CaptureStderr() }\ImportTok{as}\NormalTok{ cs:}
\NormalTok{    function\_that\_writes\_to\_stderr()}
\BuiltInTok{print}\NormalTok{(cs.err)}
\end{Highlighting}
\end{Shaded}

If you need to capture both streams at once, use the parent
\texttt{CaptureStd} class:

\begin{Shaded}
\begin{Highlighting}[]
\ImportTok{from}\NormalTok{ testing\_utils }\ImportTok{import}\NormalTok{ CaptureStd}

\ControlFlowTok{with}\NormalTok{ CaptureStd() }\ImportTok{as}\NormalTok{ cs:}
\NormalTok{    function\_that\_writes\_to\_stdout\_and\_stderr()}
\BuiltInTok{print}\NormalTok{(cs.err, cs.out)}
\end{Highlighting}
\end{Shaded}

Also, to aid debugging test issues, by default these context managers
automatically replay the captured streams on exit from the context.

\subsubsection{Capturing logger stream}\label{capturing-logger-stream}

If you need to validate the output of a logger, you can use
\texttt{CaptureLogger}:

\begin{Shaded}
\begin{Highlighting}[]
\ImportTok{from}\NormalTok{ transformers }\ImportTok{import}\NormalTok{ logging}
\ImportTok{from}\NormalTok{ testing\_utils }\ImportTok{import}\NormalTok{ CaptureLogger}

\NormalTok{msg }\OperatorTok{=} \StringTok{"Testing 1, 2, 3"}
\NormalTok{logging.set\_verbosity\_info()}
\NormalTok{logger }\OperatorTok{=}\NormalTok{ logging.get\_logger(}\StringTok{"transformers.models.bart.tokenization\_bart"}\NormalTok{)}
\ControlFlowTok{with}\NormalTok{ CaptureLogger(logger) }\ImportTok{as}\NormalTok{ cl:}
\NormalTok{    logger.info(msg)}
\ControlFlowTok{assert}\NormalTok{ cl.out, msg }\OperatorTok{+} \StringTok{"}\CharTok{\textbackslash{}n}\StringTok{"}
\end{Highlighting}
\end{Shaded}

\subsection{Testing with environment
variables}\label{testing-with-environment-variables}

If you want to test the impact of environment variables for a specific
test you can use a helper decorator
\texttt{transformers.testing\_utils.mockenv}

\begin{Shaded}
\begin{Highlighting}[]
\ImportTok{from}\NormalTok{ testing\_utils }\ImportTok{import}\NormalTok{ mockenv}


\KeywordTok{class}\NormalTok{ HfArgumentParserTest(unittest.TestCase):}
    \AttributeTok{@mockenv}\NormalTok{(TRANSFORMERS\_VERBOSITY}\OperatorTok{=}\StringTok{"error"}\NormalTok{)}
    \KeywordTok{def}\NormalTok{ test\_env\_override(}\VariableTok{self}\NormalTok{):}
\NormalTok{        env\_level\_str }\OperatorTok{=}\NormalTok{ os.getenv(}\StringTok{"TRANSFORMERS\_VERBOSITY"}\NormalTok{, }\VariableTok{None}\NormalTok{)}
\end{Highlighting}
\end{Shaded}

At times an external program needs to be called, which requires setting
\texttt{PYTHONPATH} in \texttt{os.environ} to include multiple local
paths. A helper class \texttt{testing\_utils.TestCasePlus} comes to
help:

\begin{Shaded}
\begin{Highlighting}[]
\ImportTok{from}\NormalTok{ testing\_utils }\ImportTok{import}\NormalTok{ TestCasePlus}


\KeywordTok{class}\NormalTok{ EnvExampleTest(TestCasePlus):}
    \KeywordTok{def}\NormalTok{ test\_external\_prog(}\VariableTok{self}\NormalTok{):}
\NormalTok{        env }\OperatorTok{=} \VariableTok{self}\NormalTok{.get\_env()}
        \CommentTok{\# now call the external program, passing \textasciigrave{}env\textasciigrave{} to it}
\end{Highlighting}
\end{Shaded}

Depending on whether the test file was under the \texttt{tests} test
suite or \texttt{examples} it'll correctly set up
\texttt{env{[}PYTHONPATH{]}} to include one of these two directories,
and also the \texttt{src} directory to ensure the testing is done
against the current repo, and finally with whatever
\texttt{env{[}PYTHONPATH{]}} was already set to before the test was
called if anything.

This helper method creates a copy of the \texttt{os.environ} object, so
the original remains intact.

\subsection{Getting reproducible
results}\label{getting-reproducible-results}

In some situations you may want to remove randomness for your tests. To
get identical reproducible results set, you will need to fix the seed:

\begin{Shaded}
\begin{Highlighting}[]
\NormalTok{seed }\OperatorTok{=} \DecValTok{42}

\CommentTok{\# python RNG}
\ImportTok{import}\NormalTok{ random}

\NormalTok{random.seed(seed)}

\CommentTok{\# pytorch RNGs}
\ImportTok{import}\NormalTok{ torch}

\NormalTok{torch.manual\_seed(seed)}
\NormalTok{torch.backends.cudnn.deterministic }\OperatorTok{=} \VariableTok{True}
\ControlFlowTok{if}\NormalTok{ torch.cuda.is\_available():}
\NormalTok{    torch.cuda.manual\_seed\_all(seed)}

\CommentTok{\# numpy RNG}
\ImportTok{import}\NormalTok{ numpy }\ImportTok{as}\NormalTok{ np}

\NormalTok{np.random.seed(seed)}

\CommentTok{\# tf RNG}
\NormalTok{tf.random.set\_seed(seed)}
\end{Highlighting}
\end{Shaded}

\section{Debugging tests}\label{debugging-tests}

To start a debugger at the point of the warning, do this:

\begin{Shaded}
\begin{Highlighting}[]
\ExtensionTok{pytest}\NormalTok{ tests/utils/test\_logging.py }\AttributeTok{{-}W}\NormalTok{ error::UserWarning }\AttributeTok{{-}{-}pdb}
\end{Highlighting}
\end{Shaded}

\section{A massive hack to create multiple pytest
reports}\label{a-massive-hack-to-create-multiple-pytest-reports}

Here is a massive \texttt{pytest} patching that I have done many years
ago to aid with understanding CI reports better.

To activate it add to \texttt{tests/conftest.py} (or create it if you
haven't already):

\begin{Shaded}
\begin{Highlighting}[]
\ImportTok{import}\NormalTok{ pytest}

\KeywordTok{def}\NormalTok{ pytest\_addoption(parser):}
    \ImportTok{from}\NormalTok{ testing\_utils }\ImportTok{import}\NormalTok{ pytest\_addoption\_shared}

\NormalTok{    pytest\_addoption\_shared(parser)}


\KeywordTok{def}\NormalTok{ pytest\_terminal\_summary(terminalreporter):}
    \ImportTok{from}\NormalTok{ testing\_utils }\ImportTok{import}\NormalTok{ pytest\_terminal\_summary\_main}

\NormalTok{    make\_reports }\OperatorTok{=}\NormalTok{ terminalreporter.config.getoption(}\StringTok{"{-}{-}make{-}reports"}\NormalTok{)}
    \ControlFlowTok{if}\NormalTok{ make\_reports:}
\NormalTok{        pytest\_terminal\_summary\_main(terminalreporter, }\BuiltInTok{id}\OperatorTok{=}\NormalTok{make\_reports)}
\end{Highlighting}
\end{Shaded}

and then when you run the test suite, add
\texttt{-\/-make-reports=mytests} like so:

\begin{Shaded}
\begin{Highlighting}[]
\ExtensionTok{pytest} \AttributeTok{{-}{-}make{-}reports}\OperatorTok{=}\NormalTok{mytests tests}
\end{Highlighting}
\end{Shaded}

and it'll create 8 separate reports:

\begin{Shaded}
\begin{Highlighting}[]
\ExtensionTok{$}\NormalTok{ ls }\AttributeTok{{-}1}\NormalTok{ reports/mytests/}
\ExtensionTok{durations.txt}
\ExtensionTok{errors.txt}
\ExtensionTok{failures\_line.txt}
\ExtensionTok{failures\_long.txt}
\ExtensionTok{failures\_short.txt}
\ExtensionTok{stats.txt}
\ExtensionTok{summary\_short.txt}
\ExtensionTok{warnings.txt}
\end{Highlighting}
\end{Shaded}

so now instead of having only a single output from \texttt{pytest} with
everything together, you can now have each type of report saved into
each own file.

This feature is most useful on CI, which makes it much easier to both
introspect problems and also view and download individual reports.

Using a different value to \texttt{-\/-make-reports=} for different
groups of tests can have each group saved separately rather than
clobbering each other.

All this functionality was already inside \texttt{pytest} but there was
no way to extract it easily so I added the monkey-patching overrides
\url{testing_utils.py}. Well, I did ask if I can contribute this as a
feature to \texttt{pytest} but my proposal wasn't welcome.



\end{document}
