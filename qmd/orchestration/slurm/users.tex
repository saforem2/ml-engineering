% Options for packages loaded elsewhere
\PassOptionsToPackage{unicode}{hyperref}
\PassOptionsToPackage{hyphens}{url}
\PassOptionsToPackage{dvipsnames,svgnames,x11names}{xcolor}
%
\documentclass[
]{report}

\usepackage{amsmath,amssymb}
\usepackage{iftex}
\ifPDFTeX
  \usepackage[T1]{fontenc}
  \usepackage[utf8]{inputenc}
  \usepackage{textcomp} % provide euro and other symbols
\else % if luatex or xetex
  \usepackage{unicode-math}
  \defaultfontfeatures{Scale=MatchLowercase}
  \defaultfontfeatures[\rmfamily]{Ligatures=TeX,Scale=1}
\fi
\usepackage{lmodern}
\ifPDFTeX\else  
    % xetex/luatex font selection
\fi
% Use upquote if available, for straight quotes in verbatim environments
\IfFileExists{upquote.sty}{\usepackage{upquote}}{}
\IfFileExists{microtype.sty}{% use microtype if available
  \usepackage[]{microtype}
  \UseMicrotypeSet[protrusion]{basicmath} % disable protrusion for tt fonts
}{}
\makeatletter
\@ifundefined{KOMAClassName}{% if non-KOMA class
  \IfFileExists{parskip.sty}{%
    \usepackage{parskip}
  }{% else
    \setlength{\parindent}{0pt}
    \setlength{\parskip}{6pt plus 2pt minus 1pt}}
}{% if KOMA class
  \KOMAoptions{parskip=half}}
\makeatother
\usepackage{xcolor}
\setlength{\emergencystretch}{3em} % prevent overfull lines
\setcounter{secnumdepth}{-\maxdimen} % remove section numbering
% Make \paragraph and \subparagraph free-standing
\ifx\paragraph\undefined\else
  \let\oldparagraph\paragraph
  \renewcommand{\paragraph}[1]{\oldparagraph{#1}\mbox{}}
\fi
\ifx\subparagraph\undefined\else
  \let\oldsubparagraph\subparagraph
  \renewcommand{\subparagraph}[1]{\oldsubparagraph{#1}\mbox{}}
\fi

\usepackage{color}
\usepackage{fancyvrb}
\newcommand{\VerbBar}{|}
\newcommand{\VERB}{\Verb[commandchars=\\\{\}]}
\DefineVerbatimEnvironment{Highlighting}{Verbatim}{commandchars=\\\{\}}
% Add ',fontsize=\small' for more characters per line
\usepackage{framed}
\definecolor{shadecolor}{RGB}{241,243,245}
\newenvironment{Shaded}{\begin{snugshade}}{\end{snugshade}}
\newcommand{\AlertTok}[1]{\textcolor[rgb]{0.68,0.00,0.00}{#1}}
\newcommand{\AnnotationTok}[1]{\textcolor[rgb]{0.37,0.37,0.37}{#1}}
\newcommand{\AttributeTok}[1]{\textcolor[rgb]{0.40,0.45,0.13}{#1}}
\newcommand{\BaseNTok}[1]{\textcolor[rgb]{0.68,0.00,0.00}{#1}}
\newcommand{\BuiltInTok}[1]{\textcolor[rgb]{0.00,0.23,0.31}{#1}}
\newcommand{\CharTok}[1]{\textcolor[rgb]{0.13,0.47,0.30}{#1}}
\newcommand{\CommentTok}[1]{\textcolor[rgb]{0.37,0.37,0.37}{#1}}
\newcommand{\CommentVarTok}[1]{\textcolor[rgb]{0.37,0.37,0.37}{\textit{#1}}}
\newcommand{\ConstantTok}[1]{\textcolor[rgb]{0.56,0.35,0.01}{#1}}
\newcommand{\ControlFlowTok}[1]{\textcolor[rgb]{0.00,0.23,0.31}{#1}}
\newcommand{\DataTypeTok}[1]{\textcolor[rgb]{0.68,0.00,0.00}{#1}}
\newcommand{\DecValTok}[1]{\textcolor[rgb]{0.68,0.00,0.00}{#1}}
\newcommand{\DocumentationTok}[1]{\textcolor[rgb]{0.37,0.37,0.37}{\textit{#1}}}
\newcommand{\ErrorTok}[1]{\textcolor[rgb]{0.68,0.00,0.00}{#1}}
\newcommand{\ExtensionTok}[1]{\textcolor[rgb]{0.00,0.23,0.31}{#1}}
\newcommand{\FloatTok}[1]{\textcolor[rgb]{0.68,0.00,0.00}{#1}}
\newcommand{\FunctionTok}[1]{\textcolor[rgb]{0.28,0.35,0.67}{#1}}
\newcommand{\ImportTok}[1]{\textcolor[rgb]{0.00,0.46,0.62}{#1}}
\newcommand{\InformationTok}[1]{\textcolor[rgb]{0.37,0.37,0.37}{#1}}
\newcommand{\KeywordTok}[1]{\textcolor[rgb]{0.00,0.23,0.31}{#1}}
\newcommand{\NormalTok}[1]{\textcolor[rgb]{0.00,0.23,0.31}{#1}}
\newcommand{\OperatorTok}[1]{\textcolor[rgb]{0.37,0.37,0.37}{#1}}
\newcommand{\OtherTok}[1]{\textcolor[rgb]{0.00,0.23,0.31}{#1}}
\newcommand{\PreprocessorTok}[1]{\textcolor[rgb]{0.68,0.00,0.00}{#1}}
\newcommand{\RegionMarkerTok}[1]{\textcolor[rgb]{0.00,0.23,0.31}{#1}}
\newcommand{\SpecialCharTok}[1]{\textcolor[rgb]{0.37,0.37,0.37}{#1}}
\newcommand{\SpecialStringTok}[1]{\textcolor[rgb]{0.13,0.47,0.30}{#1}}
\newcommand{\StringTok}[1]{\textcolor[rgb]{0.13,0.47,0.30}{#1}}
\newcommand{\VariableTok}[1]{\textcolor[rgb]{0.07,0.07,0.07}{#1}}
\newcommand{\VerbatimStringTok}[1]{\textcolor[rgb]{0.13,0.47,0.30}{#1}}
\newcommand{\WarningTok}[1]{\textcolor[rgb]{0.37,0.37,0.37}{\textit{#1}}}

\providecommand{\tightlist}{%
  \setlength{\itemsep}{0pt}\setlength{\parskip}{0pt}}\usepackage{longtable,booktabs,array}
\usepackage{calc} % for calculating minipage widths
% Correct order of tables after \paragraph or \subparagraph
\usepackage{etoolbox}
\makeatletter
\patchcmd\longtable{\par}{\if@noskipsec\mbox{}\fi\par}{}{}
\makeatother
% Allow footnotes in longtable head/foot
\IfFileExists{footnotehyper.sty}{\usepackage{footnotehyper}}{\usepackage{footnote}}
\makesavenoteenv{longtable}
\usepackage{graphicx}
\makeatletter
\def\maxwidth{\ifdim\Gin@nat@width>\linewidth\linewidth\else\Gin@nat@width\fi}
\def\maxheight{\ifdim\Gin@nat@height>\textheight\textheight\else\Gin@nat@height\fi}
\makeatother
% Scale images if necessary, so that they will not overflow the page
% margins by default, and it is still possible to overwrite the defaults
% using explicit options in \includegraphics[width, height, ...]{}
\setkeys{Gin}{width=\maxwidth,height=\maxheight,keepaspectratio}
% Set default figure placement to htbp
\makeatletter
\def\fps@figure{htbp}
\makeatother

\makeatletter
\@ifpackageloaded{caption}{}{\usepackage{caption}}
\AtBeginDocument{%
\ifdefined\contentsname
  \renewcommand*\contentsname{Table of contents}
\else
  \newcommand\contentsname{Table of contents}
\fi
\ifdefined\listfigurename
  \renewcommand*\listfigurename{List of Figures}
\else
  \newcommand\listfigurename{List of Figures}
\fi
\ifdefined\listtablename
  \renewcommand*\listtablename{List of Tables}
\else
  \newcommand\listtablename{List of Tables}
\fi
\ifdefined\figurename
  \renewcommand*\figurename{Figure}
\else
  \newcommand\figurename{Figure}
\fi
\ifdefined\tablename
  \renewcommand*\tablename{Table}
\else
  \newcommand\tablename{Table}
\fi
}
\@ifpackageloaded{float}{}{\usepackage{float}}
\floatstyle{ruled}
\@ifundefined{c@chapter}{\newfloat{codelisting}{h}{lop}}{\newfloat{codelisting}{h}{lop}[chapter]}
\floatname{codelisting}{Listing}
\newcommand*\listoflistings{\listof{codelisting}{List of Listings}}
\makeatother
\makeatletter
\makeatother
\makeatletter
\@ifpackageloaded{caption}{}{\usepackage{caption}}
\@ifpackageloaded{subcaption}{}{\usepackage{subcaption}}
\makeatother
\ifLuaTeX
  \usepackage{selnolig}  % disable illegal ligatures
\fi
\usepackage{bookmark}

\IfFileExists{xurl.sty}{\usepackage{xurl}}{} % add URL line breaks if available
\urlstyle{same} % disable monospaced font for URLs
\hypersetup{
  colorlinks=true,
  linkcolor={blue},
  filecolor={Maroon},
  citecolor={Blue},
  urlcolor={Blue},
  pdfcreator={LaTeX via pandoc}}

\author{}
\date{2024-02-20}

\begin{document}

\chapter{SLURM for users}\label{slurm-for-users}

\section{Quick start}\label{quick-start}

Simply copy this \href{./example.slurm}{example.slurm} and adapt it to
your needs.

\section{SLURM partitions}\label{slurm-partitions}

In this doc we will use an example setup with these 2 cluster names:

\begin{itemize}
\tightlist
\item
  \texttt{dev}
\item
  \texttt{prod}
\end{itemize}

To find out the hostname of the nodes and their availability, use:

\begin{Shaded}
\begin{Highlighting}[]
\ExtensionTok{sinfo} \AttributeTok{{-}p}\NormalTok{ dev}
\ExtensionTok{sinfo} \AttributeTok{{-}p}\NormalTok{ prod}
\end{Highlighting}
\end{Shaded}

Slurm configuration is at \texttt{/opt/slurm/etc/slurm.conf}.

\section{Wait time for resource
granting}\label{wait-time-for-resource-granting}

\begin{Shaded}
\begin{Highlighting}[]
\ExtensionTok{squeue} \AttributeTok{{-}u} \KeywordTok{\textasciigrave{}}\FunctionTok{whoami}\KeywordTok{\textasciigrave{}} \AttributeTok{{-}{-}start}
\end{Highlighting}
\end{Shaded}

will show when any pending jobs are scheduled to start.

They may start sooner if others cancel their reservations before the end
of the reservation.

\section{Request allocation via
dependency}\label{request-allocation-via-dependency}

To schedule a new job when one more of the currently scheduled job ends
(regardless of whether it still running or not started yet), use the
dependency mechanism, by telling \texttt{sbatch} to start the new job
once the currently running job succeeds, using:

\begin{Shaded}
\begin{Highlighting}[]
\ExtensionTok{sbatch} \AttributeTok{{-}{-}dependency}\OperatorTok{=}\NormalTok{CURRENTLY\_RUNNING\_JOB\_ID tr1{-}13B{-}round1.slurm}
\end{Highlighting}
\end{Shaded}

Using \texttt{-\/-dependency} may lead to shorter wait times that using
\texttt{-\/-begin}, since if the time passed to \texttt{-\/-begin}
allows even for a few minutes of delay since the stopping of the last
job, the scheduler may already start some other jobs even if their
priority is lower than our job. That's because the scheduler ignores any
jobs with \texttt{-\/-begin} until the specified time arrives.

\section{Make allocations at a scheduled
time}\label{make-allocations-at-a-scheduled-time}

To postpone making the allocation for a given time, use:

\begin{Shaded}
\begin{Highlighting}[]
\ExtensionTok{salloc} \AttributeTok{{-}{-}begin}\NormalTok{ HH:MM MM/DD/YY}
\end{Highlighting}
\end{Shaded}

Same for \texttt{sbatch}.

It will simply put the job into the queue at the requested time, as if
you were to execute this command at this time. If resources are
available at that time, the allocation will be given right away.
Otherwise it'll be queued up.

Sometimes the relative begin time is useful. And other formats can be
used. Examples:

\begin{Shaded}
\begin{Highlighting}[]
\ExtensionTok{{-}{-}begin}\NormalTok{ now+2hours}
\ExtensionTok{{-}{-}begin=16:00}
\ExtensionTok{{-}{-}begin=now+1hour}
\ExtensionTok{{-}{-}begin=now+60}  \CommentTok{\# seconds by default}
\ExtensionTok{{-}{-}begin=2010{-}01{-}20T12:34:00}
\end{Highlighting}
\end{Shaded}

the time-units can be \texttt{seconds} (default), \texttt{minutes},
\texttt{hours}, \texttt{days}, or \texttt{weeks}:

\section{Preallocated node without time 60min
limit}\label{preallocated-node-without-time-60min-limit}

This is very useful for running repetitive interactive experiments - so
one doesn't need to wait for an allocation to progress. so the strategy
is to allocate the resources once for an extended period of time and
then running interactive \texttt{srun} jobs using this allocation.

set \texttt{-\/-time} to the desired window (e.g.~6h):

\begin{Shaded}
\begin{Highlighting}[]
\ExtensionTok{salloc} \AttributeTok{{-}{-}partition}\OperatorTok{=}\NormalTok{dev }\AttributeTok{{-}{-}nodes}\OperatorTok{=}\NormalTok{1 }\AttributeTok{{-}{-}ntasks{-}per{-}node}\OperatorTok{=}\NormalTok{1 }\AttributeTok{{-}{-}cpus{-}per{-}task}\OperatorTok{=}\NormalTok{96 }\AttributeTok{{-}{-}gres}\OperatorTok{=}\NormalTok{gpu:8 }\AttributeTok{{-}{-}time}\OperatorTok{=}\NormalTok{6:00:00 bash}
\ExtensionTok{salloc:}\NormalTok{ Pending job allocation 1732778}
\ExtensionTok{salloc:}\NormalTok{ job 1732778 queued and waiting for resources}
\ExtensionTok{salloc:}\NormalTok{ job 1732778 has been allocated resources}
\ExtensionTok{salloc:}\NormalTok{ Granted job allocation 1732778}
\end{Highlighting}
\end{Shaded}

now use this reserved node to run a job multiple times, by passing the
job id of \texttt{salloc}:

\begin{Shaded}
\begin{Highlighting}[]
\ExtensionTok{srun} \AttributeTok{{-}{-}jobid} \VariableTok{$SLURM\_JOBID} \AttributeTok{{-}{-}pty}\NormalTok{ bash}
\end{Highlighting}
\end{Shaded}

if run from inside \texttt{bash} started via \texttt{salloc}. But it can
be started from another shell, but then explicitly set
\texttt{-\/-jobid}.

if this \texttt{srun} job timed out or manually exited, you can re-start
it again in this same reserved node.

\texttt{srun} can, of course, call the real training command directly
and not just \texttt{bash}.

Important: when allocating a single node, the allocated shell is not on
the node (it never is). You have to find out the hostname of the node
(reports when giving the allocation or via \texttt{squeue} and
\texttt{ssh} to it.

When finished, to release the resources, either exit the shell started
in \texttt{salloc} or \texttt{scancel\ JOBID}.

This reserved node will be counted towards hours usage the whole time
it's allocated, so release as soon as done with it.

Actually, if this is just one node, then it's even easier to not use
\texttt{salloc} but to use \texttt{srun} in the first place, which will
both allocate and give you the shell to use:

\begin{Shaded}
\begin{Highlighting}[]
\ExtensionTok{srun} \AttributeTok{{-}{-}pty} \AttributeTok{{-}{-}partition}\OperatorTok{=}\NormalTok{dev }\AttributeTok{{-}{-}nodes}\OperatorTok{=}\NormalTok{1 }\AttributeTok{{-}{-}ntasks}\OperatorTok{=}\NormalTok{1 }\AttributeTok{{-}{-}cpus{-}per{-}task}\OperatorTok{=}\NormalTok{96 }\AttributeTok{{-}{-}gres}\OperatorTok{=}\NormalTok{gpu:8 }\AttributeTok{{-}{-}time}\OperatorTok{=}\NormalTok{60 bash}
\end{Highlighting}
\end{Shaded}

\section{Hyper-Threads}\label{hyper-threads}

By default, if the cpu has hyper-threads (HT), SLURM will use it. If you
don't want to use HT you have to specify
\texttt{-\/-hint=nomultithread}.

footnote: HT is Intel-specific naming, the general concept is
simultaneous multithreading (SMT)

For example for a cluster with with 2 cpus per node with 24 cores and 2
hyper-threads each, there is a total of 96 hyper-threads or 48 cpu-cores
available. Therefore to utilize the node fully you'd need to configure
either:

\begin{Shaded}
\begin{Highlighting}[]
\CommentTok{\#SBATCH {-}{-}cpus{-}per{-}task=96}
\end{Highlighting}
\end{Shaded}

or if you don't want HT:

\begin{Shaded}
\begin{Highlighting}[]
\CommentTok{\#SBATCH {-}{-}cpus{-}per{-}task=48}
\CommentTok{\#SBATCH {-}{-}hint=nomultithread}
\end{Highlighting}
\end{Shaded}

This last approach will allocate one thread per core and in this mode
there are only 48 cpu cores to use.

Note that depending on your application there can be quite a performance
difference between these 2 modes. Therefore try both and see which one
gives you a better outcome.

On some setups like AWS the all-reduce throughput degrades dramatically
when \texttt{-\/-hint=nomultithread} is used! Whereas on some other
setups the opposite is true - the throughput is worse without HT!

\section{Reuse allocation}\label{reuse-allocation}

e.g.~when wanting to run various jobs on identical node allocation.

In one shell:

\begin{Shaded}
\begin{Highlighting}[]
\ExtensionTok{salloc} \AttributeTok{{-}{-}partition}\OperatorTok{=}\NormalTok{prod }\AttributeTok{{-}{-}nodes}\OperatorTok{=}\NormalTok{16 }\AttributeTok{{-}{-}ntasks}\OperatorTok{=}\NormalTok{16 }\AttributeTok{{-}{-}cpus{-}per{-}task}\OperatorTok{=}\NormalTok{96 }\AttributeTok{{-}{-}gres}\OperatorTok{=}\NormalTok{gpu:8 }\AttributeTok{{-}{-}time}\OperatorTok{=}\NormalTok{3:00:00 bash}
\BuiltInTok{echo} \VariableTok{$SLURM\_JOBID}
\end{Highlighting}
\end{Shaded}

In another shell:

\begin{Shaded}
\begin{Highlighting}[]
\BuiltInTok{export} \VariableTok{SLURM\_JOBID}\OperatorTok{=\textless{}}\NormalTok{JOB ID FROM ABOVE}\OperatorTok{\textgreater{}}
\ExtensionTok{srun} \AttributeTok{{-}{-}jobid}\OperatorTok{=}\VariableTok{$SLURM\_JOBID}\NormalTok{ ...}
\end{Highlighting}
\end{Shaded}

You may need to set \texttt{-\/-gres=gpu:0} to run some diagnostics job
on the nodes. For example, let's check shared memory of all the hosts:

\begin{Shaded}
\begin{Highlighting}[]
\ExtensionTok{srun} \AttributeTok{{-}{-}jobid}\NormalTok{ 631078 }\AttributeTok{{-}{-}gres}\OperatorTok{=}\NormalTok{gpu:0 bash }\AttributeTok{{-}c} \StringTok{\textquotesingle{}echo $(hostname) $(df {-}h | grep shm)\textquotesingle{}}
\end{Highlighting}
\end{Shaded}

\section{Specific nodes selection}\label{specific-nodes-selection}

To exclude specific nodes (useful when you know some nodes are broken,
but are still in IDLE state):

\begin{Shaded}
\begin{Highlighting}[]
\ExtensionTok{sbatch} \AttributeTok{{-}{-}exclude}\NormalTok{ nodeA,nodeB}
\end{Highlighting}
\end{Shaded}

or via: \texttt{\#SBATCH\ -\/-exclude\ ...}

To use specific nodes:

\begin{Shaded}
\begin{Highlighting}[]
\ExtensionTok{sbatch} \AttributeTok{{-}{-}nodelist}\OperatorTok{=}\NormalTok{ nodeA,nodeB}
\end{Highlighting}
\end{Shaded}

can also use the short \texttt{-w} instead of \texttt{-\/-nodelist}

The administrator could also define a \texttt{feature=example} in
\texttt{slurm.conf} and then a user could ask for that subset of nodes
via \texttt{-\/-constraint=example}

\section{Signal the running jobs to
finish}\label{signal-the-running-jobs-to-finish}

Since each SLURM run has a limited time span, it can be configured to
send a signal of choice to the program a desired amount of time before
the end of the allocated time.

\begin{Shaded}
\begin{Highlighting}[]
\ExtensionTok{{-}{-}signal=[[R][B]:]}\OperatorTok{\textless{}}\NormalTok{sig\_num}\OperatorTok{\textgreater{}}\NormalTok{[@}\OperatorTok{\textless{}}\NormalTok{sig\_time}\OperatorTok{\textgreater{}}\NormalTok{]}
\end{Highlighting}
\end{Shaded}

TODO: need to experiment with this to help training finish gracefully
and not start a new cycle after saving the last checkpoint.

\section{Detailed job info}\label{detailed-job-info}

While most useful information is preset in various \texttt{SLURM\_*} env
vars, sometimes the info is missing. In such cases use:

\begin{Shaded}
\begin{Highlighting}[]
\ExtensionTok{scontrol}\NormalTok{ show }\AttributeTok{{-}d}\NormalTok{ job }\VariableTok{$SLURM\_JOB\_ID}
\end{Highlighting}
\end{Shaded}

and then parse out what's needed.

For a job that finished its run use:

\begin{Shaded}
\begin{Highlighting}[]
\ExtensionTok{sacct} \AttributeTok{{-}j}\NormalTok{ JOBID}
\end{Highlighting}
\end{Shaded}

e.g.~with more details, depending on the partition:

\begin{Shaded}
\begin{Highlighting}[]
\ExtensionTok{sacct} \AttributeTok{{-}u} \KeywordTok{\textasciigrave{}}\FunctionTok{whoami}\KeywordTok{\textasciigrave{}} \AttributeTok{{-}{-}partition}\OperatorTok{=}\NormalTok{dev  }\AttributeTok{{-}ojobid,start,end,state,exitcode} \AttributeTok{{-}{-}format}\NormalTok{ nodelist\%300  }\AttributeTok{{-}j}\NormalTok{ JOBID}
\ExtensionTok{sacct} \AttributeTok{{-}u} \KeywordTok{\textasciigrave{}}\FunctionTok{whoami}\KeywordTok{\textasciigrave{}} \AttributeTok{{-}{-}partition}\OperatorTok{=}\NormalTok{prod }\AttributeTok{{-}ojobid,start,end,state,exitcode} \AttributeTok{{-}{-}format}\NormalTok{ nodelist\%300  }\AttributeTok{{-}j}\NormalTok{ JOBID}
\end{Highlighting}
\end{Shaded}

\section{show jobs}\label{show-jobs}

Show only my jobs:

\begin{Shaded}
\begin{Highlighting}[]
\ExtensionTok{squeue} \AttributeTok{{-}u} \KeywordTok{\textasciigrave{}}\FunctionTok{whoami}\KeywordTok{\textasciigrave{}}
\end{Highlighting}
\end{Shaded}

Show jobs by job id:

\begin{Shaded}
\begin{Highlighting}[]
\ExtensionTok{squeue} \AttributeTok{{-}j}\NormalTok{ JOBID}
\end{Highlighting}
\end{Shaded}

Show jobs of a specific partition:

\begin{Shaded}
\begin{Highlighting}[]
\ExtensionTok{squeue} \AttributeTok{{-}{-}partition}\OperatorTok{=}\NormalTok{dev}
\end{Highlighting}
\end{Shaded}

\section{Aliases}\label{aliases}

Handy aliases

\begin{Shaded}
\begin{Highlighting}[]
\BuiltInTok{alias}\NormalTok{ myjobs=}\StringTok{\textquotesingle{}squeue {-}u \textasciigrave{}whoami\textasciigrave{} {-}o "\%.16i \%9P \%26j \%.8T \%.10M \%.8l \%.6D \%.20S \%R"\textquotesingle{}}
\BuiltInTok{alias}\NormalTok{ groupjobs=}\StringTok{\textquotesingle{}squeue {-}u foo,bar,tar {-}o "\%.16i \%u \%9P \%26j \%.8T \%.10M \%.8l \%.6D \%.20S \%R"\textquotesingle{}}
\BuiltInTok{alias}\NormalTok{ myjobs{-}pending=}\StringTok{"squeue {-}u }\KeywordTok{\textasciigrave{}}\FunctionTok{whoami}\KeywordTok{\textasciigrave{}}\StringTok{ {-}{-}start"}
\BuiltInTok{alias}\NormalTok{ idle{-}nodes=}\StringTok{"sinfo {-}p prod {-}o \textquotesingle{}\%A\textquotesingle{}"}
\end{Highlighting}
\end{Shaded}

\section{Zombies}\label{zombies}

If there are any zombies left behind across nodes, send one command to
kill them all.

\begin{Shaded}
\begin{Highlighting}[]
\ExtensionTok{srun}\NormalTok{ pkill python}
\end{Highlighting}
\end{Shaded}

\section{Detailed Access to SLURM
Accounting}\label{detailed-access-to-slurm-accounting}

\texttt{sacct} displays accounting data for all jobs and job steps in
the Slurm job accounting log or Slurm database.

So this is a great tool for analysing past events.

For example, to see which nodes were used to run recent gpu jobs:

\begin{Shaded}
\begin{Highlighting}[]
\ExtensionTok{sacct} \AttributeTok{{-}u} \KeywordTok{\textasciigrave{}}\FunctionTok{whoami}\KeywordTok{\textasciigrave{}} \AttributeTok{{-}{-}partition}\OperatorTok{=}\NormalTok{dev }\AttributeTok{{-}ojobid,start,end,state,exitcode} \AttributeTok{{-}{-}format}\NormalTok{ nodelist\%300}
\end{Highlighting}
\end{Shaded}

\texttt{\%300} here tells it to use a 300 char width for the output, so
that it's not truncated.

See \texttt{man\ sacct} for more fields and info fields.

\section{Queue}\label{queue}

\subsection{Cancel job}\label{cancel-job}

To cancel a job:

\begin{Shaded}
\begin{Highlighting}[]
\ExtensionTok{scancel} \PreprocessorTok{[}\SpecialStringTok{jobid}\PreprocessorTok{]}
\end{Highlighting}
\end{Shaded}

To cancel all of your jobs:

\begin{Shaded}
\begin{Highlighting}[]
\ExtensionTok{scancel} \AttributeTok{{-}u} \OperatorTok{\textless{}}\NormalTok{userid}\OperatorTok{\textgreater{}}
\end{Highlighting}
\end{Shaded}

To cancel all of your jobs on a specific partition:

\begin{Shaded}
\begin{Highlighting}[]
\ExtensionTok{scancel} \AttributeTok{{-}u} \OperatorTok{\textless{}}\NormalTok{userid}\OperatorTok{\textgreater{}}\NormalTok{ {-}p }\OperatorTok{\textless{}}\NormalTok{partition}\OperatorTok{\textgreater{}}
\end{Highlighting}
\end{Shaded}

\subsection{Tips}\label{tips}

\begin{itemize}
\tightlist
\item
  if you see that \texttt{salloc}'ed interactive job is scheduled to run
  much later than you need, try to cancel the job and ask for shorter
  period - often there might be a closer window for a shorter time
  allocation.
\end{itemize}

\section{Logging}\label{logging}

If we need to separate logs to different log files per node add
\texttt{\%N} (for short hostname) so that we have:

\begin{Shaded}
\begin{Highlighting}[]
\CommentTok{\#SBATCH {-}{-}output=\%x{-}\%j{-}\%N.out}
\end{Highlighting}
\end{Shaded}

That way we can tell if a specific node misbehaves - e.g.~has a corrupt
GPU. This is because currently pytorch doesn't log which node / gpu rank
triggered an exception.

Hoping it'll be a built-in feature of pytorch
https://github.com/pytorch/pytorch/issues/63174 and then one won't need
to make things complicated on the logging side.

\section{Show the state of nodes}\label{show-the-state-of-nodes}

\begin{Shaded}
\begin{Highlighting}[]
\ExtensionTok{sinfo} \AttributeTok{{-}p}\NormalTok{ PARTITION}
\end{Highlighting}
\end{Shaded}

Very useful command is:

\begin{Shaded}
\begin{Highlighting}[]
\ExtensionTok{sinfo} \AttributeTok{{-}s}
\end{Highlighting}
\end{Shaded}

and look for the main stat, e.g.:

\begin{Shaded}
\begin{Highlighting}[]
\ExtensionTok{NODES}\ErrorTok{(}\ExtensionTok{A/I/O/T}\KeywordTok{)} \StringTok{"allocated/idle/other/total"}\ExtensionTok{.}
\ExtensionTok{597/0/15/612}
\end{Highlighting}
\end{Shaded}

So here 597 out of 612 nodes are allocated. 0 idle and 15 are not
available for whatever other reasons.

\begin{Shaded}
\begin{Highlighting}[]
\ExtensionTok{sinfo} \AttributeTok{{-}p}\NormalTok{ gpu\_p1 }\AttributeTok{{-}o} \StringTok{"\%A"}
\end{Highlighting}
\end{Shaded}

gives:

\begin{Shaded}
\begin{Highlighting}[]
\ExtensionTok{NODES}\ErrorTok{(}\ExtensionTok{A/I}\KeywordTok{)}
\ExtensionTok{236/24}
\end{Highlighting}
\end{Shaded}

so you can see if any nodes are available on the 4x v100-32g partition
(\texttt{gpu\_p1})

To check a specific partition:

\begin{Shaded}
\begin{Highlighting}[]
\ExtensionTok{sinfo} \AttributeTok{{-}p}\NormalTok{ gpu\_p1 }\AttributeTok{{-}o} \StringTok{"\%A"}
\end{Highlighting}
\end{Shaded}

See the table at the top of this document for which partition is which.

\subsection{sinfo states}\label{sinfo-states}

\begin{itemize}
\tightlist
\item
  idle: no jobs running
\item
  alloc: nodes are allocated to jobs that are currently executing
\item
  mix: the nodes have some of the CPUs allocated, while others are idle
\item
  drain: the node is unavailable due to an administrative reason
\item
  drng: the node is running a job, but will after completion not be
  available due to an administrative reason
\end{itemize}

\subsection{drained nodes}\label{drained-nodes}

To see all drained nodes and the reason for drainage (edit
\texttt{\%50E} to make the reason field longer/shorter)

\begin{Shaded}
\begin{Highlighting}[]
\ExtensionTok{\%}\NormalTok{ sinfo }\AttributeTok{{-}R} \AttributeTok{{-}o} \StringTok{"\%50E \%12U \%19H \%6t \%N"}
\end{Highlighting}
\end{Shaded}

or just \texttt{-R} if you want it short:

\begin{Shaded}
\begin{Highlighting}[]
\ExtensionTok{\%}\NormalTok{ sinfo }\AttributeTok{{-}R}
\end{Highlighting}
\end{Shaded}

\section{Job arrays}\label{job-arrays}

To run a sequence of jobs, so that the next slurm job is scheduled as
soon as the currently running one is over in 20h we use a job array.

Let's start with just 10 such jobs:

\begin{Shaded}
\begin{Highlighting}[]
\ExtensionTok{sbatch} \AttributeTok{{-}{-}array}\OperatorTok{=}\NormalTok{1{-}10\%1 array{-}test.slurm}
\end{Highlighting}
\end{Shaded}

\texttt{\%1} limits the number of simultaneously running tasks from this
job array to 1. Without it it will try to run all the jobs at once,
which we may want sometimes (in which case remove \%1), but when
training we need one job at a time.

Alternatively, as always this param can be part of the script:

\begin{Shaded}
\begin{Highlighting}[]
\CommentTok{\#SBATCH {-}{-}array=1{-}10\%1}
\end{Highlighting}
\end{Shaded}

Here is toy slurm script, which can be used to see how it works:

\begin{Shaded}
\begin{Highlighting}[]
\CommentTok{\#!/bin/bash}
\CommentTok{\#SBATCH {-}{-}job{-}name=array{-}test}
\CommentTok{\#SBATCH {-}{-}nodes=1}
\CommentTok{\#SBATCH {-}{-}ntasks{-}per{-}node=1          \# crucial {-} only 1 task per dist per node!}
\CommentTok{\#SBATCH {-}{-}cpus{-}per{-}task=1            \# number of cores per tasks}
\CommentTok{\#SBATCH {-}{-}time 00:02:00              \# maximum execution time (HH:MM:SS)}
\CommentTok{\#SBATCH {-}{-}output=\%x{-}\%j.out           \# output file name}
\CommentTok{\#SBATCH {-}{-}error=\%x{-}\%j.out            \# error file name (same to watch just one file)}
\CommentTok{\#SBATCH {-}{-}partition=dev}

\BuiltInTok{echo} \VariableTok{$SLURM\_JOB\_ID}
\BuiltInTok{echo} \StringTok{"I am job }\VariableTok{$\{SLURM\_ARRAY\_JOB\_ID\}}\StringTok{\_}\VariableTok{$\{SLURM\_ARRAY\_TASK\_ID\}}\StringTok{"}
\FunctionTok{date}
\FunctionTok{sleep}\NormalTok{ 10}
\FunctionTok{date}
\end{Highlighting}
\end{Shaded}

Note \texttt{\$SLURM\_ARRAY\_JOB\_ID} is the same as
\texttt{\$SLURM\_JOB\_ID}, and \texttt{\$SLURM\_ARRAY\_TASK\_ID} is the
index of the job.

To see the jobs running:

\begin{Shaded}
\begin{Highlighting}[]
\ExtensionTok{$}\NormalTok{ squeue }\AttributeTok{{-}u} \KeywordTok{\textasciigrave{}}\FunctionTok{whoami}\KeywordTok{\textasciigrave{}} \AttributeTok{{-}o} \StringTok{"\%.10i \%9P \%26j \%.8T \%.10M \%.6D \%.20S \%R"}
     \ExtensionTok{JOBID}\NormalTok{ PARTITION                       NAME    STATE       TIME  NODES           START\_TIME NODELIST}\ErrorTok{(}\ExtensionTok{REASON}\KeywordTok{)}
\ExtensionTok{591970\_[2{-}}\NormalTok{   dev             array{-}test  PENDING       0:00      1  2021{-}07{-}28T20:01:06 }\ErrorTok{(}\ExtensionTok{JobArrayTaskLimit}\KeywordTok{)}
\end{Highlighting}
\end{Shaded}

now job 2 is running.

To cancel the whole array, cancel the job id as normal (the number
before \texttt{\_}):

\begin{Shaded}
\begin{Highlighting}[]
\ExtensionTok{scancel}\NormalTok{ 591970}
\end{Highlighting}
\end{Shaded}

To cancel a specific job:

\begin{Shaded}
\begin{Highlighting}[]
\ExtensionTok{scancel}\NormalTok{ 591970\_2}
\end{Highlighting}
\end{Shaded}

If it's important to have the log-file contain the array id, add
\texttt{\%A\_\%a}:

\begin{Shaded}
\begin{Highlighting}[]
\CommentTok{\#SBATCH {-}{-}output=\%x{-}\%j.\%A\_\%a.log}
\end{Highlighting}
\end{Shaded}

More details https://slurm.schedmd.com/job\_array.html

\section{Job Array Trains and their Suspend and
Release}\label{job-array-trains-and-their-suspend-and-release}

In this recipe we accomplish 2 things:

\begin{enumerate}
\def\labelenumi{\arabic{enumi}.}
\tightlist
\item
  Allow modification to the next job's slurm script
\item
  Allow suspending and resuming job arrays w/o losing the place in the
  queue when not being ready to continue running a job
\end{enumerate}

SLURM is a very unforgiving environment where a small mistake can cost
days of waiting time. But there are strategies to mitigate some of this
harshness.

SLURM jobs have a concept of ``age'' in the queue which besides project
priority governs when a job gets scheduled to run. If your have just
scheduled a new job it has no ``age'' and will normally be put to run
last compared to jobs that have entered the queue earlier. Unless of
course this new job comes from a high priority project in which case
it'll progress faster.

So here is how one can keep the ``age'' and not lose it when needing to
fix something in the running script or for example to switch over to
another script.

The idea is this:

\begin{enumerate}
\def\labelenumi{\arabic{enumi}.}
\tightlist
\item
  \texttt{sbatch} a long job array, e.g., \texttt{-array=1-50\%1}
\item
  inside the slurm script don't have any code other than
  \texttt{source\ another-script.slurm} - so now you can modify the
  target script or symlink to another script before the next job starts
\item
  if you need to stop the job array train - don't cancel it, but suspend
  it without losing your place in a queue
\item
  when ready to continue - unsuspend the job array - only the time while
  it was suspended is not counted towards its age, but all the previous
  age is retained.
\end{enumerate}

The only limitation of this recipe is that you can't change the number
of nodes, time and hardware and partition constraints once the job array
was launched.

Here is an example:

Create a job script:

\begin{Shaded}
\begin{Highlighting}[]
\ExtensionTok{$}\NormalTok{ cat train{-}64n.slurm}
\CommentTok{\#!/bin/bash}
\CommentTok{\#SBATCH {-}{-}job{-}name=tr8{-}104B}
\CommentTok{\#SBATCH {-}{-}nodes=64}
\CommentTok{\#SBATCH {-}{-}ntasks{-}per{-}node=1          \# crucial {-} only 1 task per dist per node!}
\CommentTok{\#SBATCH {-}{-}cpus{-}per{-}task=96           \# number of cores per tasks}
\CommentTok{\#SBATCH {-}{-}gres=gpu:8                 \# number of gpus}
\CommentTok{\#SBATCH {-}{-}time 20:00:00              \# maximum execution time (HH:MM:SS)}
\CommentTok{\#SBATCH {-}{-}output=\%x{-}\%j.out           \# output file name}
\CommentTok{\#SBATCH {-}{-}partition=dev}

\BuiltInTok{source}\NormalTok{ tr8{-}104B{-}64.slurm}
\end{Highlighting}
\end{Shaded}

Start it as:

\begin{Shaded}
\begin{Highlighting}[]
\ExtensionTok{sbatch} \AttributeTok{{-}{-}array}\OperatorTok{=}\NormalTok{1{-}50\%1 train{-}64.slurm}
\end{Highlighting}
\end{Shaded}

Now you can easily edit \texttt{tr8-104B-64.slurm} before the next job
run and either let the current job finish if it's desired or if you need
to abort it, just kill the currently running job,
e.g.~\texttt{1557903\_5} (not job array \texttt{1557903}) and have the
train pick up where it left, but with the edited script.

The nice thing is that this requires no changes to the original script
(\texttt{tr8-104B-64.slurm} in this example), and the latter can still
be started on its own.

Now, what if something is wrong and you need 10min or 10h to fix
something. In this case we suspend the train using:

\begin{Shaded}
\begin{Highlighting}[]
\ExtensionTok{scontrol}\NormalTok{ hold }\OperatorTok{\textless{}}\NormalTok{jobid}\OperatorTok{\textgreater{}}
\end{Highlighting}
\end{Shaded}

with being either a ``normal'' job, the id of a job array or the id for
a job array step

and then when ready to continue release the job:

\begin{Shaded}
\begin{Highlighting}[]
\ExtensionTok{scontrol}\NormalTok{ release }\OperatorTok{\textless{}}\NormalTok{jobid}\OperatorTok{\textgreater{}}
\end{Highlighting}
\end{Shaded}

\section{Troubleshooting}\label{troubleshooting}

\subsection{Mismatching nodes number}\label{mismatching-nodes-number}

If the pytorch launcher fails it often means that the number of SLURM
nodes and the launcher nodes are mismatching, e.g.:

\begin{Shaded}
\begin{Highlighting}[]
\FunctionTok{grep} \AttributeTok{{-}ir}\NormalTok{ nodes= tr123{-}test.slurm}
\CommentTok{\#SBATCH {-}{-}nodes=40}
\VariableTok{NNODES}\OperatorTok{=}\NormalTok{64}
\end{Highlighting}
\end{Shaded}

This won't work. They have to match.

You can add a sanity check to your script:

\begin{Shaded}
\begin{Highlighting}[]
\CommentTok{\#!/bin/bash}
\CommentTok{\#SBATCH {-}{-}job{-}name=test{-}mismatch}
\CommentTok{\#SBATCH {-}{-}nodes=2}
\CommentTok{\#SBATCH {-}{-}ntasks{-}per{-}node=1          \# crucial {-} only 1 task per dist per node!}
\CommentTok{\#SBATCH {-}{-}cpus{-}per{-}task=96           \# number of cores per tasks}
\CommentTok{\#SBATCH {-}{-}gres=gpu:8                 \# number of gpus}
\CommentTok{\#SBATCH {-}{-}time 0:05:00               \# maximum execution time (HH:MM:SS)}
\CommentTok{\#SBATCH {-}{-}output=\%x{-}\%j.out           \# output file name}
\CommentTok{\#SBATCH {-}{-}partition=prod}

\ExtensionTok{[...]}

\VariableTok{NNODES}\OperatorTok{=}\NormalTok{2}

\CommentTok{\# sanity check for having NNODES and \textasciigrave{}\#SBATCH {-}{-}nodes\textasciigrave{} match, assuming you use NNODES variable}
\ControlFlowTok{if} \BuiltInTok{[} \StringTok{"}\VariableTok{$NNODES}\StringTok{"} \OtherTok{!=} \StringTok{"}\VariableTok{$SLURM\_NNODES}\StringTok{"} \BuiltInTok{]}\KeywordTok{;} \ControlFlowTok{then}
    \BuiltInTok{echo} \StringTok{"Misconfigured script: NNODES=}\VariableTok{$NNODES}\StringTok{ != SLURM\_NNODES=}\VariableTok{$SLURM\_NNODES}\StringTok{"}
    \BuiltInTok{exit}\NormalTok{ 1}
\ControlFlowTok{fi}

\ExtensionTok{[...]}
\end{Highlighting}
\end{Shaded}

or you could just do:

\begin{Shaded}
\begin{Highlighting}[]
\CommentTok{\#SBATCH {-}{-}nodes=2}
\ExtensionTok{[...]}
\VariableTok{NNODES}\OperatorTok{=}\VariableTok{$SLURM\_NNODES}
\end{Highlighting}
\end{Shaded}

and then it will always be correct

\subsection{Find faulty nodes and exclude
them}\label{find-faulty-nodes-and-exclude-them}

Sometimes a node is broken, which prevents one from training, especially
since restarting the job often hits the same set of nodes. So one needs
to be able to isolate the bad node(s) and exclude it from
\texttt{sbatch}.

To find a faulty node, write a small script that reports back the status
of the desired check.

For example to test if cuda is available on all nodes:

\begin{Shaded}
\begin{Highlighting}[]
\ExtensionTok{python} \AttributeTok{{-}c} \StringTok{\textquotesingle{}import torch, socket; print(f"\{socket.gethostname()\}: \{torch.cuda.is\_available()\}")\textquotesingle{}}
\end{Highlighting}
\end{Shaded}

and to only report the nodes that fail:

\begin{Shaded}
\begin{Highlighting}[]
\ExtensionTok{python} \AttributeTok{{-}c} \StringTok{\textquotesingle{}import torch, socket; torch.cuda.is\_available() or print(f"Broken node: \{socket.gethostname()\}") \textquotesingle{}}
\end{Highlighting}
\end{Shaded}

Of course, the issue could be different - e.g.~gpu can't allocate
memory, so change the test script to do a small allocation on cuda. Here
is one way:

\begin{Shaded}
\begin{Highlighting}[]
\ExtensionTok{python} \AttributeTok{{-}c} \StringTok{"import torch; torch.ones(1000,1000).cuda()"}
\end{Highlighting}
\end{Shaded}

But since we need to run the test script on all nodes and not just the
first node, the slurm script needs to run it via \texttt{srun}. So our
first diagnostics script can be written as:

\begin{Shaded}
\begin{Highlighting}[]
\ExtensionTok{srun} \AttributeTok{{-}{-}jobid} \VariableTok{$SLURM\_JOBID}\NormalTok{ bash }\AttributeTok{{-}c} \StringTok{\textquotesingle{}python {-}c "import torch, socket; print(socket.gethostname(), torch.cuda.is\_available())"\textquotesingle{}}
\end{Highlighting}
\end{Shaded}

I slightly changed it, due to an issue with quotes.

You can always convert the one liner into a real script and then there
is no issue with quotes.

\begin{Shaded}
\begin{Highlighting}[]
\ExtensionTok{$}\NormalTok{ cat }\OperatorTok{\textless{}\textless{} EOT} \OperatorTok{\textgreater{}\textgreater{}}\NormalTok{ test{-}nodes.py}
\StringTok{\#!/usr/bin/env python}
\StringTok{import torch, socket}
\StringTok{print(socket.gethostname(), torch.cuda.is\_available())}
\OperatorTok{EOT}
\ExtensionTok{$}\NormalTok{ chmod a+x ./test{-}nodes.py}
\end{Highlighting}
\end{Shaded}

Now let's create a driver slurm script. Use a few minutes time for this
test so that SLURM yields it faster:

\begin{Shaded}
\begin{Highlighting}[]
\CommentTok{\#!/bin/bash}
\CommentTok{\#SBATCH {-}{-}job{-}name=test{-}nodes}
\CommentTok{\#SBATCH {-}{-}nodes=4}
\CommentTok{\#SBATCH {-}{-}ntasks{-}per{-}node=1          \# crucial {-} only 1 task per dist per node!}
\CommentTok{\#SBATCH {-}{-}cpus{-}per{-}task=96           \# number of cores per tasks}
\CommentTok{\#SBATCH {-}{-}gres=gpu:8                 \# number of gpus}
\CommentTok{\#SBATCH {-}{-}time 0:05:00               \# maximum execution time (HH:MM:SS)}
\CommentTok{\#SBATCH {-}{-}output=\%x{-}\%j.out           \# output file name}
\CommentTok{\#SBATCH {-}{-}partition=prod}

\BuiltInTok{source} \VariableTok{$six\_ALL\_CCFRWORK}\NormalTok{/start{-}prod}
\ExtensionTok{srun} \AttributeTok{{-}{-}jobid} \VariableTok{$SLURM\_JOBID}\NormalTok{ ./test{-}nodes.py}
\end{Highlighting}
\end{Shaded}

Once it runs check the logs to see if any reported \texttt{False}, those
are the nodes you want to exclude.

Now once the faulty node(s) is found, feed it to \texttt{sbatch}:

\begin{Shaded}
\begin{Highlighting}[]
\ExtensionTok{sbatch} \AttributeTok{{-}{-}exclude}\OperatorTok{=}\NormalTok{hostname1,hostname2 ...}
\end{Highlighting}
\end{Shaded}

and \texttt{sbatch} will exclude the bad nodes from the allocation.

Additionally please report the faulty nodes to
\texttt{\#science-support} so that they get replaced

Here are a few more situations and how to find the bad nodes in those
cases:

\subsection{Broken NCCL}\label{broken-nccl}

If you're testing something that requires distributed setup, it's a bit
more complex. Here is a slurm script that tests that NCCL works. It sets
up NCCL and checks that barrier works:

\begin{Shaded}
\begin{Highlighting}[]
\CommentTok{\#!/bin/bash}
\CommentTok{\#SBATCH {-}{-}job{-}name=test{-}nodes{-}nccl}
\CommentTok{\#SBATCH {-}{-}nodes=2}
\CommentTok{\#SBATCH {-}{-}ntasks{-}per{-}node=1          \# crucial {-} only 1 task per dist per node!}
\CommentTok{\#SBATCH {-}{-}cpus{-}per{-}task=96           \# number of cores per tasks}
\CommentTok{\#SBATCH {-}{-}gres=gpu:8                 \# number of gpus}
\CommentTok{\#SBATCH {-}{-}time 0:05:00               \# maximum execution time (HH:MM:SS)}
\CommentTok{\#SBATCH {-}{-}output=\%x{-}\%j.out           \# output file name}
\CommentTok{\#SBATCH {-}{-}partition=prod}

\BuiltInTok{source} \VariableTok{$six\_ALL\_CCFRWORK}\NormalTok{/start{-}prod}

\VariableTok{NNODES}\OperatorTok{=}\NormalTok{2}

\VariableTok{GPUS\_PER\_NODE}\OperatorTok{=}\NormalTok{4}
\VariableTok{MASTER\_ADDR}\OperatorTok{=}\VariableTok{$(}\ExtensionTok{scontrol}\NormalTok{ show hostnames }\VariableTok{$SLURM\_JOB\_NODELIST} \KeywordTok{|} \FunctionTok{head} \AttributeTok{{-}n}\NormalTok{ 1}\VariableTok{)}
\VariableTok{MASTER\_PORT}\OperatorTok{=}\NormalTok{6000}

\BuiltInTok{export} \VariableTok{LAUNCHER}\OperatorTok{=}\StringTok{"python {-}u {-}m torch.distributed.launch }\DataTypeTok{\textbackslash{}}
\StringTok{    {-}{-}nproc\_per\_node }\VariableTok{$GPUS\_PER\_NODE}\StringTok{ }\DataTypeTok{\textbackslash{}}
\StringTok{    {-}{-}nnodes }\VariableTok{$NNODES}\StringTok{ }\DataTypeTok{\textbackslash{}}
\StringTok{    {-}{-}master\_addr }\VariableTok{$MASTER\_ADDR}\StringTok{ }\DataTypeTok{\textbackslash{}}
\StringTok{    {-}{-}master\_port }\VariableTok{$MASTER\_PORT}\StringTok{ }\DataTypeTok{\textbackslash{}}
\StringTok{    "}

\BuiltInTok{export} \VariableTok{SCRIPT}\OperatorTok{=}\NormalTok{test{-}nodes{-}nccl.py}

\FunctionTok{cat} \OperatorTok{\textless{}\textless{} EOT} \OperatorTok{\textgreater{}} \VariableTok{$SCRIPT}
\StringTok{\#!/usr/bin/env python}
\StringTok{import torch.distributed as dist}
\StringTok{import torch}
\StringTok{import socket}
\StringTok{import os}
\StringTok{import fcntl}

\StringTok{def printflock(*msgs):}
\StringTok{    """ print """}
\StringTok{    with open(\_\_file\_\_, "r") as fh:}
\StringTok{        fcntl.flock(fh, fcntl.LOCK\_EX)}
\StringTok{        try:}
\StringTok{            print(*msgs)}
\StringTok{        finally:}
\StringTok{            fcntl.flock(fh, fcntl.LOCK\_UN)}

\StringTok{local\_rank = int(os.environ["LOCAL\_RANK"])}
\StringTok{torch.cuda.set\_device(local\_rank)}
\StringTok{dist.init\_process\_group("nccl")}
\StringTok{header = f"\{socket.gethostname()\}{-}\{local\_rank\}"}
\StringTok{try:}
\StringTok{    dist.barrier()}
\StringTok{    printflock(f"\{header\}: NCCL \{torch.cuda.nccl.version()\} is OK")}
\StringTok{except:}
\StringTok{    printflock(f"\{header\}: NCCL \{torch.cuda.nccl.version()\} is broken")}
\StringTok{    raise}
\OperatorTok{EOT}

\BuiltInTok{echo} \VariableTok{$LAUNCHER} \AttributeTok{{-}{-}node\_rank} \VariableTok{$SLURM\_PROCID} \VariableTok{$SCRIPT}

\ExtensionTok{srun} \AttributeTok{{-}{-}jobid} \VariableTok{$SLURM\_JOBID}\NormalTok{ bash }\AttributeTok{{-}c} \StringTok{\textquotesingle{}$LAUNCHER {-}{-}node\_rank $SLURM\_PROCID $SCRIPT\textquotesingle{}}
\end{Highlighting}
\end{Shaded}

The script uses \texttt{printflock} to solve the interleaved print
outputs issue.

\subsection{GPU Memory Check}\label{gpu-memory-check}

This tests if each GPU on the allocated nodes can successfully allocate
77Gb (e.g.~to test 80GB A100s) (have to subtract a few GBs for cuda
kernels).

\begin{Shaded}
\begin{Highlighting}[]
\ImportTok{import}\NormalTok{ torch, os}
\ImportTok{import}\NormalTok{ time}
\ImportTok{import}\NormalTok{ socket}
\NormalTok{hostname }\OperatorTok{=}\NormalTok{ socket.gethostname()}

\NormalTok{local\_rank }\OperatorTok{=} \BuiltInTok{int}\NormalTok{(os.environ[}\StringTok{"LOCAL\_RANK"}\NormalTok{])}\OperatorTok{;}

\NormalTok{gbs }\OperatorTok{=} \DecValTok{77}
\ControlFlowTok{try}\NormalTok{:}
\NormalTok{    torch.ones((gbs}\OperatorTok{*}\DecValTok{2}\OperatorTok{**}\DecValTok{28}\NormalTok{)).cuda(local\_rank).contiguous() }\CommentTok{\# alloc on cpu, then move to gpu}
    \BuiltInTok{print}\NormalTok{(}\SpecialStringTok{f"}\SpecialCharTok{\{}\NormalTok{local\_rank}\SpecialCharTok{\}}\SpecialStringTok{ }\SpecialCharTok{\{}\NormalTok{hostname}\SpecialCharTok{\}}\SpecialStringTok{ is OK"}\NormalTok{)}
\ControlFlowTok{except}\NormalTok{:}
    \BuiltInTok{print}\NormalTok{(}\SpecialStringTok{f"}\SpecialCharTok{\{}\NormalTok{local\_rank}\SpecialCharTok{\}}\SpecialStringTok{ }\SpecialCharTok{\{}\NormalTok{hostname}\SpecialCharTok{\}}\SpecialStringTok{ failed to allocate }\SpecialCharTok{\{}\NormalTok{gbs}\SpecialCharTok{\}}\SpecialStringTok{GB DRAM"}\NormalTok{)}
    \ControlFlowTok{pass}

\NormalTok{time.sleep(}\DecValTok{5}\NormalTok{)}
\end{Highlighting}
\end{Shaded}

\subsection{Broken Network}\label{broken-network}

Yet another issue with a node is when its network is broken and other
nodes fail to connect to it.

You're likely to experience it with an error similar to:

\begin{Shaded}
\begin{Highlighting}[]
\ExtensionTok{work}\NormalTok{ = default\_pg.barrier}\ErrorTok{(}\VariableTok{opts}\OperatorTok{=}\NormalTok{opts}\KeywordTok{)}
\ExtensionTok{RuntimeError:}\NormalTok{ NCCL error in: /opt/conda/conda{-}bld/pytorch\_1616554793803/work/torch/lib/c10d/ProcessGroupNCCL.cpp:825, unhandled system error, NCCL version 2.7.8}
\ExtensionTok{ncclSystemError:}\NormalTok{ System call }\ErrorTok{(}\ExtensionTok{socket,}\NormalTok{ malloc, munmap, etc}\KeywordTok{)} \ExtensionTok{failed.}
\end{Highlighting}
\end{Shaded}

Here is how to debug this issue:

\begin{enumerate}
\def\labelenumi{\arabic{enumi}.}
\item
  Add:

\begin{Shaded}
\begin{Highlighting}[]
\BuiltInTok{export} \VariableTok{NCCL\_DEBUG}\OperatorTok{=}\NormalTok{INFO}
\end{Highlighting}
\end{Shaded}

  before the \texttt{srun} command and re-run your slurm script.
\item
  Now study the logs. If you find:

\begin{Shaded}
\begin{Highlighting}[]
\ExtensionTok{r11i6n2:486514:486651} \PreprocessorTok{[}\SpecialStringTok{1}\PreprocessorTok{]}\NormalTok{ include/socket.h:403 NCCL WARN Connect to 10.148.3.247}\OperatorTok{\textless{}}\NormalTok{56821}\OperatorTok{\textgreater{}}\NormalTok{ failed : Connection refused}
\end{Highlighting}
\end{Shaded}

  Let's see which node refuses to accept connections. We get the IP
  address from the error above and reverse resolve it to its name:

\begin{Shaded}
\begin{Highlighting}[]
\ExtensionTok{nslookup}\NormalTok{ 10.148.3.247}
\ExtensionTok{247.3.148.10.in{-}addr.arpa}\NormalTok{       name = r10i6n5.ib0.xa.idris.fr.}
\end{Highlighting}
\end{Shaded}

  Add \texttt{-\/-exclude=r10i6n5} to your \texttt{sbatch} command and
  report it to JZ admins.
\end{enumerate}

\subsection{Run py-spy or any other monitor program across all
nodes}\label{run-py-spy-or-any-other-monitor-program-across-all-nodes}

When dealing with hanging, here is how to automatically log
\texttt{py-spy} traces for each process.

Of course, this same process can be used to run some command for all
nodes of a given job. i.e.~it can be used to run something during the
normal run - e.g.~dump all the memory usage in each process via
\texttt{nvidia-smi} or whatever other program is needed to be run.

\begin{Shaded}
\begin{Highlighting}[]
\BuiltInTok{cd}\NormalTok{ \textasciitilde{}/prod/code/tr8b{-}104B/bigscience/train/tr11{-}200B{-}ml/}

\ExtensionTok{salloc} \AttributeTok{{-}{-}partition}\OperatorTok{=}\NormalTok{prod }\AttributeTok{{-}{-}nodes}\OperatorTok{=}\NormalTok{40 }\AttributeTok{{-}{-}ntasks{-}per{-}node}\OperatorTok{=}\NormalTok{1 }\AttributeTok{{-}{-}cpus{-}per{-}task}\OperatorTok{=}\NormalTok{96 }\AttributeTok{{-}{-}gres}\OperatorTok{=}\NormalTok{gpu:8 }\AttributeTok{{-}{-}time}\NormalTok{ 20:00:00}

\FunctionTok{bash}\NormalTok{ 200B{-}n40{-}bf16{-}mono.slurm}
\end{Highlighting}
\end{Shaded}

In another shell get the JOBID for the above \texttt{salloc}:

\begin{Shaded}
\begin{Highlighting}[]
\ExtensionTok{squeue} \AttributeTok{{-}u} \KeywordTok{\textasciigrave{}}\FunctionTok{whoami}\KeywordTok{\textasciigrave{}} \AttributeTok{{-}o} \StringTok{"\%.16i \%9P \%26j \%.8T \%.10M \%.8l \%.6D \%.20S \%R"}
\end{Highlighting}
\end{Shaded}

adjust jobid per above and the nodes count (XXX: probably can remove
\texttt{-\/-nodes=40} altogether and rely on \texttt{salloc} config):

\begin{Shaded}
\begin{Highlighting}[]
\ExtensionTok{srun} \AttributeTok{{-}{-}jobid}\OperatorTok{=}\NormalTok{2180718 }\AttributeTok{{-}{-}gres}\OperatorTok{=}\NormalTok{gpu:0 }\AttributeTok{{-}{-}nodes}\OperatorTok{=}\NormalTok{40 }\AttributeTok{{-}{-}tasks{-}per{-}node}\OperatorTok{=}\NormalTok{1 }\AttributeTok{{-}{-}output}\OperatorTok{=}\NormalTok{trace{-}\%N.out sh }\AttributeTok{{-}c} \StringTok{\textquotesingle{}ps aux | grep python | egrep {-}v "grep|srun" | grep \textasciigrave{}whoami\textasciigrave{} | awk "\{print \textbackslash{}$2\}" | xargs {-}I \{\} py{-}spy dump {-}{-}native {-}{-}pid \{\}\textquotesingle{}} \KeywordTok{||} \BuiltInTok{echo} \StringTok{"failed"}
\end{Highlighting}
\end{Shaded}

now all \texttt{py-spy} traces go into the \texttt{trace-\$nodename.out}
files under \texttt{cwd}.

The key is to use \texttt{-\/-gres=gpu:0} or otherwise the 2nd
\texttt{srun} will block waiting for the first one to release the gpus.

Also the assumption is that some conda env that has \texttt{py-spy}
installed got activated in \texttt{\textasciitilde{}/.bashrc}. If yours
doesn't already do that, add the instruction to load the env to the
above command, before the \texttt{py-spy} command - it'll fail to find
it otherwise.

Don't forget to manually release the allocation when this process is
done.

\section{Convert SLURM\_JOB\_NODELIST into a
hostfile}\label{convert-slurm_job_nodelist-into-a-hostfile}

Some multi-node launchers require a \texttt{hostfile} - here is how to
generate one:

\begin{Shaded}
\begin{Highlighting}[]
\CommentTok{\# autogenerate the hostfile for deepspeed}
\CommentTok{\# 1. deals with: SLURM\_JOB\_NODELIST in either of 2 formats:}
\CommentTok{\# r10i1n8,r10i2n0}
\CommentTok{\# r10i1n[7{-}8]}
\CommentTok{\# 2. and relies on SLURM\_STEP\_GPUS=0,1,2... to get how many gpu slots per node}
\CommentTok{\#}
\CommentTok{\# usage:}
\CommentTok{\# makehostfile \textgreater{} hostfile}
\KeywordTok{function}\FunctionTok{ makehostfile()} \KeywordTok{\{}
\FunctionTok{perl} \AttributeTok{{-}le} \StringTok{\textquotesingle{}$slots=split /,/, $ENV\{"SLURM\_STEP\_GPUS"\}; $\_=$ENV\{"SLURM\_JOB\_NODELIST"\}; if (/\^{}(.*?)\textbackslash{}[(\textbackslash{}d+){-}(\textbackslash{}d+)\textbackslash{}]/) \{ print map \{ "$1$\_ slots=$slots\textbackslash{}n" \} $2..$3\} elsif (/,/) \{ print map \{ "$1$\_ slots=$slots\textbackslash{}n" \} split /,/ \} \textquotesingle{}}
\KeywordTok{\}}
\end{Highlighting}
\end{Shaded}

\section{Environment variables}\label{environment-variables}

You can always do:

\begin{Shaded}
\begin{Highlighting}[]
\BuiltInTok{export} \VariableTok{SOMEKEY}\OperatorTok{=}\NormalTok{value}
\end{Highlighting}
\end{Shaded}

from the slurm script to get a desired environment variable passed to
the program launched from it.

And you can also add to the top of the slurm script:

\begin{Shaded}
\begin{Highlighting}[]
\CommentTok{\#SBATCH {-}{-}export=ALL}
\end{Highlighting}
\end{Shaded}

The launched program will see all the environment variables visible in
the shell where it was launched from.

\section{Crontab Emulation}\label{crontab-emulation}

One of the most important Unix tools is the crontab, which is essential
for being able to schedule various jobs. It however usually is absent
from SLURM environment. Therefore one must emulate it. Here is how.

For this presentation we are going to use \texttt{\$WORK/cron/} as the
base directory. And that you have an exported environment variable
\texttt{WORK} pointing to some location on your filesystem - if you use
Bash you can set it up in your \texttt{\textasciitilde{}/.bash\_profile}
or if a different shell is used use whatever startup equivalent file is.

\subsection{1. A self-perpetuating scheduler
job}\label{a-self-perpetuating-scheduler-job}

We will use \texttt{\$WORK/cron/scheduler} dir for scheduler jobs,
\texttt{\$WORK/cron/cron.daily} for daily jobs and
\texttt{\$WORK/cron/cron.hourly} for hourly jobs:

\begin{Shaded}
\begin{Highlighting}[]
\ExtensionTok{$}\NormalTok{ mkdir }\AttributeTok{{-}p} \VariableTok{$WORK}\NormalTok{/cron/scheduler}
\ExtensionTok{$}\NormalTok{ mkdir }\AttributeTok{{-}p} \VariableTok{$WORK}\NormalTok{/cron/cron.daily}
\ExtensionTok{$}\NormalTok{ mkdir }\AttributeTok{{-}p} \VariableTok{$WORK}\NormalTok{/cron/cron.hourly}
\end{Highlighting}
\end{Shaded}

Now copy these two slurm script in \texttt{\$WORK/cron/scheduler}: -
\url{cron-daily.slurm} - \url{cron-hourly.slurm}

after editing those to fit your specific environment's account and
partition information.

Now you can launch the crontab scheduler jobs:

\begin{Shaded}
\begin{Highlighting}[]
\ExtensionTok{$}\NormalTok{ cd }\VariableTok{$WORK}\NormalTok{/cron/scheduler}
\ExtensionTok{$}\NormalTok{ sbatch cron{-}hourly.slurm}
\ExtensionTok{$}\NormalTok{ sbatch cron{-}daily.slurm}
\end{Highlighting}
\end{Shaded}

This is it, these jobs will now self-perpetuate and usually you don't
need to think about it again unless there is an even that makes SLURM
lose all its jobs.

\subsection{2. Daily and Hourly
Cronjobs}\label{daily-and-hourly-cronjobs}

Now whenever you want some job to run once a day, you simply create a
slurm job and put it into the \texttt{\$WORK/cron/cron.daily} dir.

Here is an example job that runs daily to update the \texttt{mlocate}
file index:

\begin{Shaded}
\begin{Highlighting}[]
\ExtensionTok{$}\NormalTok{ cat }\VariableTok{$WORK}\NormalTok{/cron/cron.daily/mlocate{-}update.slurm}
\CommentTok{\#!/bin/bash}
\CommentTok{\#SBATCH {-}{-}job{-}name=mlocate{-}update    \# job name}
\CommentTok{\#SBATCH {-}{-}ntasks=1                   \# number of MP tasks}
\CommentTok{\#SBATCH {-}{-}nodes=1}
\CommentTok{\#SBATCH {-}{-}hint=nomultithread         \# we get physical cores not logical}
\CommentTok{\#SBATCH {-}{-}time=1:00:00               \# maximum execution time (HH:MM:SS)}
\CommentTok{\#SBATCH {-}{-}output=\%x{-}\%j.out           \# output file name}
\CommentTok{\#SBATCH {-}{-}partition=PARTITION     \# edit me}
\CommentTok{\#SBATCH {-}{-}account=GROUP@PARTITION \# edit me}

\BuiltInTok{set} \AttributeTok{{-}e}
\FunctionTok{date}
\BuiltInTok{echo} \StringTok{"updating mlocate db"}
\ExtensionTok{/usr/bin/updatedb} \AttributeTok{{-}o} \VariableTok{$WORK}\NormalTok{/lib/mlocate/work.db }\AttributeTok{{-}U} \VariableTok{$WORK} \AttributeTok{{-}{-}require{-}visibility}\NormalTok{ 0}
\end{Highlighting}
\end{Shaded}

This builds an index of the files under \texttt{\$WORK} which you can
then quickly query with:

\begin{Shaded}
\begin{Highlighting}[]
\ExtensionTok{/usr/bin/locate} \AttributeTok{{-}d} \VariableTok{$WORK}\NormalTok{/lib/mlocate/work.db pattern}
\end{Highlighting}
\end{Shaded}

To stop running this job, just move it out of the
\texttt{\$WORK/cron/cron.daily} dir.

The same principle applies to jobs placed into the
\texttt{\$WORK/cron/cron.hourly} dir. These are useful for running
something every hour.

Please note that this crontab implementation is approximate timing-wise,
due to various delays in SLURM scheduling they will run approximately
every hour and every day. You can recode these to ask SLURM to start
something at a more precise time if you have to, but most of the time
the just presented method works fine.

Additionally, you can code your own variations to meet specific needs of
your project, e.g., every-30min or every-12h jobs.

\subsection{3. Cleanup}\label{cleanup}

Finally, since every cron launcher job will leave behind a log file
(which is useful if for some reason things don't work), you want to
create a cronjob to clean up these logs. Otherwise you may run out of
inodes - these logs files are tiny, but there could be tens of thousands
of those.

You could use something like this in a daily job.

\begin{Shaded}
\begin{Highlighting}[]
\FunctionTok{find} \VariableTok{$WORK}\NormalTok{/cron }\AttributeTok{{-}name} \StringTok{"*.out"} \AttributeTok{{-}mtime}\NormalTok{ +7 }\AttributeTok{{-}exec}\NormalTok{ rm }\AttributeTok{{-}f}\NormalTok{ \{\} +}
\end{Highlighting}
\end{Shaded}

Please note that it's set to only delete files that are older than 7
days, in case you need the latest logs for diagnostics.

\subsection{Nuances}\label{nuances}

The scheduler runs with Unix permissions of the person who launched the
SLRUM cron scheduler job and so all other SLURM scripts launched by that
cron job.

\section{Self-perpetuating SLURM
jobs}\label{self-perpetuating-slurm-jobs}

The same approach used in
\hyperref[1-a-self-perpetuating-scheduler-job]{building a scheduler} can
be used for creating stand-alone self-perpetuating jobs.

For example:

\begin{Shaded}
\begin{Highlighting}[]
\CommentTok{\#!/bin/bash}
\CommentTok{\#SBATCH {-}{-}job{-}name=watchdog          \# job name}
\CommentTok{\#SBATCH {-}{-}ntasks=1                   \# number of MP tasks}
\CommentTok{\#SBATCH {-}{-}nodes=1}
\CommentTok{\#SBATCH {-}{-}time=0:30:00               \# maximum execution time (HH:MM:SS)}
\CommentTok{\#SBATCH {-}{-}output=\%x{-}\%j.out           \# output file name}
\CommentTok{\#SBATCH {-}{-}partition=PARTITION        \# edit me}

\CommentTok{\# ensure to restart self first 1h from now}
\VariableTok{RUN\_FREQUENCY\_IN\_HOURS}\OperatorTok{=}\NormalTok{1}
\ExtensionTok{sbatch} \AttributeTok{{-}{-}begin}\OperatorTok{=}\NormalTok{now+}\VariableTok{$\{RUN\_FREQUENCY\_IN\_HOURS\}}\NormalTok{hour watchdog.slurm}

\ExtensionTok{...}\NormalTok{ do the watchdog work here ...}
\end{Highlighting}
\end{Shaded}

and you launch it once with:

\begin{Shaded}
\begin{Highlighting}[]
\ExtensionTok{sbatch}\NormalTok{ watchdog.slurm}
\end{Highlighting}
\end{Shaded}

This then will immediately schedule itself to be run 1 hour from the
launch time and then the normal job work will be done. Regardless of
whether the rest of the job will succeed or fail, this job will continue
relaunching itself approximately once an hour. This is imprecise due to
scheduler job starting overhead and node availability issues. But if
there is a least one spare node available and the job itself is quick to
finish the requirement to run at an approximate frequency should be
sufficient.

As the majority of SLURM environment in addition to the expensive GPU
nodes also provide much cheaper CPU-only nodes, you should choose a
CPU-only SLURM partition for any jobs that don't require GPUs to run.

\section{Getting information about the
job}\label{getting-information-about-the-job}

From within the slurm file one can access information about the current
job's allocations.

Getting allocated hostnames and useful derivations based on that:

\begin{Shaded}
\begin{Highlighting}[]
\BuiltInTok{export} \VariableTok{HOSTNAMES}\OperatorTok{=}\VariableTok{$(}\ExtensionTok{scontrol}\NormalTok{ show hostnames }\StringTok{"}\VariableTok{$SLURM\_JOB\_NODELIST}\StringTok{"}\VariableTok{)}
\BuiltInTok{export} \VariableTok{NUM\_NODES}\OperatorTok{=}\VariableTok{$(}\ExtensionTok{scontrol}\NormalTok{ show hostnames }\StringTok{"}\VariableTok{$SLURM\_JOB\_NODELIST}\StringTok{"} \KeywordTok{|} \FunctionTok{wc} \AttributeTok{{-}l}\VariableTok{)}
\BuiltInTok{export} \VariableTok{MASTER\_ADDR}\OperatorTok{=}\VariableTok{$(}\ExtensionTok{scontrol}\NormalTok{ show hostnames }\StringTok{"}\VariableTok{$SLURM\_JOB\_NODELIST}\StringTok{"} \KeywordTok{|} \FunctionTok{head} \AttributeTok{{-}n}\NormalTok{ 1}\VariableTok{)}
\end{Highlighting}
\end{Shaded}

\section{Convert compact node list to expanded node
list}\label{convert-compact-node-list-to-expanded-node-list}

Sometimes you get SLURM tools give you a string like:
\texttt{node-{[}42,49-51{]}} which will require some coding to expand it
into \texttt{node-42,node-49,node-50,node-51}, but there is a special
tool to deal with that:

\begin{Shaded}
\begin{Highlighting}[]
\ExtensionTok{$}\NormalTok{ scontrol show hostnames node{-}}\PreprocessorTok{[}\SpecialStringTok{42,49}\PreprocessorTok{{-}}\SpecialStringTok{51}\PreprocessorTok{]}
\ExtensionTok{node{-}42}
\ExtensionTok{node{-}49}
\ExtensionTok{node{-}50}
\ExtensionTok{node{-}51}
\end{Highlighting}
\end{Shaded}

Voila!

case study: this is for example useful if you want get a list of nodes
that were drained because the job was too slow to exit, but really there
is no real problem with the nodes. So this one-liner will give you the
list of such nodes in an expanded format which you can then script to
loop over this list to undrain these nodes after perhaps checking that
the processes have died by this time:

\begin{Shaded}
\begin{Highlighting}[]
\ExtensionTok{sinfo} \AttributeTok{{-}R} \KeywordTok{|} \FunctionTok{grep} \StringTok{"Kill task failed"} \KeywordTok{|} \FunctionTok{perl} \AttributeTok{{-}lne} \StringTok{\textquotesingle{}/(node{-}.*[\textbackslash{}d\textbackslash{}]]+)/ \&\& print $1\textquotesingle{}} \KeywordTok{|} \FunctionTok{xargs} \AttributeTok{{-}n1}\NormalTok{ scontrol show hostnames}
\end{Highlighting}
\end{Shaded}

\section{Overcoming the lack of group SLURM job
ownership}\label{overcoming-the-lack-of-group-slurm-job-ownership}

SLURM runs on Unix, but surprisingly its designers haven't adopted the
concept of group ownership with regards to SLURM jobs. So if a member of
your team started an array of 10 jobs 20h each, and went on vacation -
unless you have \texttt{sudo} access you now can't do anything to stop
those jobs if something is wrong.

I'm yet to find why this is so, but so far we have been using a kill
switch workaround. You have to code it in your framework. For example,
see how it was implemented in
\href{https://github.com/bigscience-workshop/Megatron-DeepSpeed/blob/e52bdabbde3c6895aceb76c1bced295c2646121f/megatron/training.py\#L104}{Megatron-Deepspeed}
(Meg-DS). The program polls for a pre-configured at start up path on the
filesystem and if it finds a file there, it exits.

So if we start Meg-DS with
\texttt{-\/-kill-switch-path\ \$WORK/tmp/training17-kill-switch} and
then at any point we need to kill the SLURM job, we simply do:

\begin{Shaded}
\begin{Highlighting}[]
\FunctionTok{touch} \VariableTok{$WORK}\NormalTok{/tmp/training17{-}kill{-}switch}
\end{Highlighting}
\end{Shaded}

and the next time the program gets to check for this file it'll detect
the event and will exit voluntarily. If you have a job array, well, you
will have to wait until each job starts, detects the kill switch and
exits.

Of course, don't forget to remove it when you're done stopping the jobs.

\begin{Shaded}
\begin{Highlighting}[]
\FunctionTok{rm} \VariableTok{$WORK}\NormalTok{/tmp/training17{-}kill{-}switch}
\end{Highlighting}
\end{Shaded}

Now, this doesn't always work. If the job is hanging, it'll never come
to the point of checking for kill-switch and the only solution here is
to contact the sysadmins to kill the job for you. Sometimes if the
hanging is a simple case pytorch's distributed setup will typically
auto-exit after 30min of preset timeout time, but it doesn't always
work.

\textless\textless\textless\textless\textless\textless\textless{}
HEAD:qmd/orchestration/slurm/users.qmd \#\# How to gracefully exit on
SLURM job preemption

\chapter{\texorpdfstring{There are several ways to gracefully handle
time- and QoS-based SLURM pre-emption which are covered indepth in this
section:
\href{https://saforem2.github.io/ml-engineering/qmd/training/fault-tolerance/\#dealing-with-forced-resource-preemption}{Dealing
with forced job
preemption}}{There are several ways to gracefully handle time- and QoS-based SLURM pre-emption which are covered indepth in this section: Dealing with forced job preemption}}\label{there-are-several-ways-to-gracefully-handle-time--and-qos-based-slurm-pre-emption-which-are-covered-indepth-in-this-section-dealing-with-forced-job-preemption}

\section{How to gracefully exit on SLURM job
preemption}\label{how-to-gracefully-exit-on-slurm-job-preemption}

There are several ways to gracefully handle time- and QoS-based SLURM
pre-emption which are covered indepth in this section:
\href{../../training/fault-tolerance/\#dealing-with-forced-job-preemption}{Dealing
with forced job preemption}.
\textgreater\textgreater\textgreater\textgreater\textgreater\textgreater\textgreater{}
fa0c7c24bf2e39579cba33ec170ea60ef11f0b93:orchestration/slurm/users.md



\end{document}
